\documentclass{CInf_practice}

\sheet{7}{Schaltwerke}
\usepackage[gen]{eurosym}

\begin{document}
\cinftitle

\ex{Schaltwerksentwurf}{4 + 16 + 8 + 2 + 6 = 36}

\subex{Ein- \& Ausgaben}
\noindent Eingaben:
\begin{enumerate}[align=left,leftmargin=\marginparwidth]
   \item[$X_C$] 50 Cent einwerfen oder nicht
   \item[$X_E$] Euromünze einwerfen oder nicht
\end{enumerate}
Ausgaben:
\begin{enumerate}[align=left,leftmargin=\marginparwidth]
   \item[$Y_R$] 50 Cent auswerfen oder nicht
   \item[$X_G$] Getränk auswerfen oder nicht
\end{enumerate}

\subex{Zustandsgraph}

Unser Automat ist insofern etwas kundenunfreundlich, als er immer in den
Ausgangszustand geht, sollte man es irgendwie hinbekommen, gleichzeitig ein
50-Cent-Münze und ein 1-Euro-Stück in den Schlitz zu quetschen. Aber wer den
Automaten so misshandelt, verdient ohnehin keine Limo.

\usetikzlibrary{automata,positioning,arrows.meta} % Try putting this line right above the tikzpicture, I dare you.

Die Semantik der Zustände ist folgendermaßen:
\begin{center}
   \begin{tabular}{ll}
      \hline
      Zustand & Semantik \\\hline
      A & Initial, wartend \\
      B & 1\euro{} fehlt \\
      C & 50 Cent fehlen \\
      D & Getränk ausgeben, kein Rückgeld\\
      E & Getränk ausgeben, Rückgeld ausgeben\\
      \hline
   \end{tabular}
\end{center}

\begin{center}
   \hspace{-2cm}\begin{tikzpicture}[node distance=2cm,very thick,yscale=.7]
      \node[scale=.8,draw,rounded corners=5pt,fill=lightgray!50,text width=4cm] (legend) {
         \makebox[4cm]{Legende}\\
         \hrulefill \\
         An Zuständen: \makebox[4cm]{$Z/Y_R Y_G$} \\ An Übergängen: \makebox[4cm]{ $X_CX_E$ }
      };

      \tikzset{state/.append style={execute at begin node=$, execute at end node=$}}
      \begin{scope}[xshift=5cm,yshift=-7cm]
         \foreach \n[count=\x] / \ausgabe in {A/00,B/00,D/01,C/00,E/11}{
            \node[state] at (\x*360/5:4cm) (\n) {\n/\ausgabe};
         }
         \tikzset{every loop/.style={looseness=5}}
         \path[-{Latex}] (A) edge[loop above] node {11,00} ();
         \path[-{Latex}] (A) edge[bend right] node[above] {10} (B);
         \path[-{Latex}] (A) edge[bend right] node[left] {01} (C);
         \path[-{Latex}] (B) edge[loop left] node[above left] {00} ();
         \path[-{Latex}] (B) edge node[above] {11} (A);
         \path[-{Latex}] (B) edge[bend left,out=0] node[left] {01} (D);
         \path[-{Latex}] (B) edge[bend right=10] node[above right] {10} (C);
         \path[-{Latex}] (D) edge[loop below] node[below] {00} ();
         \path[-{Latex}] (D) edge[bend left] node[left] {10} (B);
         \path[-{Latex}] (D) edge[bend left,in=180] node[above] {01} (C);
         \path[-{Latex}] (C) edge[loop below] node[below] {00} ();
         \path[-{Latex}] (C) edge[bend left] node[above] {10} (D);
         \path[-{Latex}] (C) edge node[above left,pos=.8] {01} (E);
         \path[-{Latex}] (E) edge[loop right] node[right] {00} ();
         \path[-{Latex}] (E) edge[bend left] node[above left] {01} (C);
         \path[-{Latex}] (E) edge[bend right=5] node[pos=.8,above right] {10} (B);
         \path[-{Latex}] (E) edge[bend right] node[right] {11} (A);

         \path[-{Latex}] (D) edge[bend left,looseness=2,out=120,in=90] node[left] {11} (A);
         \path[-{Latex}] (C) edge[bend right,looseness=2,out=230,in=270] node[right] {11} (A);
      \end{scope}
   \end{tikzpicture}
\end{center}

Der Graph ist offensichtlich konsistent und ``vollständig'', da für jede Eingabe
genau ein Ausgang pro Knoten existiert.

\subex{}

Wir haben 5 Zustände, weshalb 3 Bit zur Kodierung nötig sind.

\begin{center}
   \begin{tabular}{ll}
      \hline
      Zustand & Kodierung \\\hline
      A & 000 \\
      B & 001 \\
      C & 010 \\
      D & 011 \\
      E & 100 \\
      \hline
   \end{tabular}
\end{center}

Die Zustandsübergangstabelle ist damit
\begin{center}
   \addtolength{\tabcolsep}{-3pt}
   \begin{tabular}{cccc|cc|cc|ccc}
      \hline
      Zustand & $Z_2^n$ & $Z_1^n$ & $Z_0^n$ & $X_C$ & $X_E$ & $Y_R$ & $Y_G$ &
      $Z_2^{n+1}$ & $Z_1^{n+1}$ & $Z_0^{n+1}$\\
      \hline\hline
      A & 0 & 0 & 0 & 0 & 0 & 0 & 0 & 0 & 0 & 0 \\
      A & 0 & 0 & 0 & 0 & 1 & 0 & 0 & 0 & 1 & 0 \\
      A & 0 & 0 & 0 & 1 & 0 & 0 & 0 & 0 & 0 & 1 \\
      A & 0 & 0 & 0 & 1 & 1 & 0 & 0 & X & X & X \\\hline

      B & 0 & 0 & 1 & 0 & 0 & 0 & 0 & 0 & 0 & 1 \\
      B & 0 & 0 & 1 & 0 & 1 & 0 & 0 & 0 & 1 & 1 \\
      B & 0 & 0 & 1 & 1 & 0 & 0 & 0 & 0 & 1 & 0 \\
      B & 0 & 0 & 1 & 1 & 1 & 0 & 0 & X & X & X \\\hline

      C & 0 & 1 & 0 & 0 & 0 & 0 & 0 & 0 & 1 & 0 \\
      C & 0 & 1 & 0 & 0 & 1 & 0 & 0 & 1 & 0 & 0 \\
      C & 0 & 1 & 0 & 1 & 0 & 0 & 0 & 0 & 1 & 1 \\
      C & 0 & 1 & 0 & 1 & 1 & 0 & 0 & X & X & X \\\hline

      D & 0 & 1 & 1 & 0 & 0 & 0 & 1 & 0 & 1 & 1 \\
      D & 0 & 1 & 1 & 0 & 1 & 0 & 1 & 0 & 1 & 0 \\
      D & 0 & 1 & 1 & 1 & 0 & 0 & 1 & 0 & 0 & 1 \\
      D & 0 & 1 & 1 & 1 & 1 & 0 & 1 & X & X & X \\\hline

      E & 1 & 0 & 0 & 0 & 0 & 1 & 1 & 1 & 0 & 0 \\
      E & 1 & 0 & 0 & 0 & 1 & 1 & 1 & 0 & 1 & 0 \\
      E & 1 & 0 & 0 & 1 & 0 & 1 & 1 & 0 & 0 & 1 \\
      E & 1 & 0 & 0 & 1 & 1 & 1 & 1 & X & X & X \\\hline

        & 1 & 0 & 1 & X & X & X & X & X & X & X \\
        & 1 & 1 & 0 & X & X & X & X & X & X & X \\
        & 1 & 1 & 1 & X & X & X & X & X & X & X \\\hline
   \end{tabular}
\end{center}

Unter Ausnutzung der Don't-Cares minimieren sich die Ausgänge zu
\begin{eqnarray*}
   Y_R & = & Z_2^n \\
   Y_G & = & Z_2^n + Z_1^nZ_0^n
\end{eqnarray*}

Da für D-Flipflops gilt $Q^{n+1} = D^n$, sind die Ansteuersignale genau die
Bits, die den Folgezustand codieren.

\begin{center}
   \addtolength{\tabcolsep}{-3pt}
   \begin{tabular}{cccc|cc|cc|ccc>{\columncolor{lightgray!50}}c>{\columncolor{lightgray!50}}c>{\columncolor{lightgray!50}}c}
      \hline
      Zustand & $Z_2^n$ & $Z_1^n$ & $Z_0^n$ & $X_C$ & $X_E$ & $Y_R$ & $Y_G$ &
      $Z_2^{n+1}$ & $Z_1^{n+1}$ & $Z_0^{n+1}$ & $D_2^n$ & $D_1^n$ & $D_0^n$ \\
      \hline\hline
      A & 0 & 0 & 0 & 0 & 0 & 0 & 0 & 0 & 0 & 0 & 0 & 0 & 0 \\
      A & 0 & 0 & 0 & 0 & 1 & 0 & 0 & 0 & 1 & 0 & 0 & 1 & 0 \\
      A & 0 & 0 & 0 & 1 & 0 & 0 & 0 & 0 & 0 & 1 & 0 & 0 & 1 \\
      A & 0 & 0 & 0 & 1 & 1 & 0 & 0 & X & X & X & X & X & X \\\hline

      B & 0 & 0 & 1 & 0 & 0 & 0 & 0 & 0 & 0 & 1 & 0 & 0 & 1 \\
      B & 0 & 0 & 1 & 0 & 1 & 0 & 0 & 0 & 1 & 1 & 0 & 1 & 1 \\
      B & 0 & 0 & 1 & 1 & 0 & 0 & 0 & 0 & 1 & 0 & 0 & 1 & 0 \\
      B & 0 & 0 & 1 & 1 & 1 & 0 & 0 & X & X & X & X & X & X \\\hline

      C & 0 & 1 & 0 & 0 & 0 & 0 & 0 & 0 & 1 & 0 & 0 & 1 & 0 \\
      C & 0 & 1 & 0 & 0 & 1 & 0 & 0 & 1 & 0 & 0 & 1 & 0 & 0 \\
      C & 0 & 1 & 0 & 1 & 0 & 0 & 0 & 0 & 1 & 1 & 0 & 0 & 1 \\
      C & 0 & 1 & 0 & 1 & 1 & 0 & 0 & X & X & X & X & X & X \\\hline

      D & 0 & 1 & 1 & 0 & 0 & 0 & 1 & 0 & 1 & 1 & 0 & 1 & 1 \\
      D & 0 & 1 & 1 & 0 & 1 & 0 & 1 & 0 & 1 & 0 & 0 & 1 & 0 \\
      D & 0 & 1 & 1 & 1 & 0 & 0 & 1 & 0 & 0 & 1 & 0 & 0 & 1 \\
      D & 0 & 1 & 1 & 1 & 1 & 0 & 1 & X & X & X & X & X & X \\\hline

      E & 1 & 0 & 0 & 0 & 0 & 1 & 1 & 1 & 0 & 0 & 1 & 0 & 0 \\
      E & 1 & 0 & 0 & 0 & 1 & 1 & 1 & 0 & 1 & 0 & 0 & 1 & 0 \\
      E & 1 & 0 & 0 & 1 & 0 & 1 & 1 & 0 & 0 & 1 & 0 & 0 & 1 \\
      E & 1 & 0 & 0 & 1 & 1 & 1 & 1 & X & X & X & X & X & X \\\hline

        & 1 & 0 & 1 & X & X & X & X & X & X & X & X & X & X \\
        & 1 & 1 & 0 & X & X & X & X & X & X & X & X & X & X \\
        & 1 & 1 & 1 & X & X & X & X & X & X & X & X & X & X \\\hline
   \end{tabular}
\end{center}
\end{document}
