\documentclass{CInf_practice}

\sheet{7}{Schaltwerke}
\usepackage[gen]{eurosym}

\begin{document}
\cinftitle

\ex{Schaltwerksentwurf}{4 + 16 + 8 + 2 + 6 = 36}

\subex{Ein- \& Ausgaben}
\noindent Eingaben:
\begin{enumerate}[align=left,leftmargin=\marginparwidth]
   \item[$X_C$] 50 Cent einwerfen oder nicht
   \item[$X_E$] Euromünze einwerfen oder nicht
\end{enumerate}
Ausgaben:
\begin{enumerate}[align=left,leftmargin=\marginparwidth]
   \item[$Y_R$] 50 Cent auswerfen oder nicht
   \item[$X_G$] Getränk auswerfen oder nicht
\end{enumerate}

\subex{Zustandsgraph}

Unser Automat ist insofern etwas kundenunfreundlich, als er immer in den
Ausgangszustand geht, sollte man es irgendwie hinbekommen, gleichzeitig ein
50-Cent-Münze und ein 1-Euro-Stück in den Schlitz zu quetschen. Aber wer den
Automaten so misshandelt, verdient ohnehin keine Limo.

\usetikzlibrary{automata,positioning,arrows.meta} % Try putting this line right above the tikzpicture, I dare you.

Die Semantik der Zustände ist folgendermaßen:
\begin{ctabular}{ll}
   \hline
   Zustand & Semantik \\\hline
   A & Initial, wartend \\
   B & 1\euro{} fehlt \\
   C & 50 Cent fehlen \\
   D & Getränk ausgeben, kein Rückgeld\\
   E & Getränk ausgeben, Rückgeld ausgeben\\
   \hline
\end{ctabular}

\begin{center}
   \hspace{-2cm}\begin{tikzpicture}[node distance=2cm,very thick,yscale=.7]
      \node[scale=.8,draw,rounded corners=5pt,fill=lightgray!50,text width=4cm] (legend) {
         \makebox[4cm]{Legende}\\
         \hrulefill \\
         An Zuständen: \makebox[4cm]{$Z/Y_R Y_G$} \\ An Übergängen: \makebox[4cm]{ $X_CX_E$ }
      };

      \tikzset{state/.append style={execute at begin node=$, execute at end node=$}}
      \begin{scope}[xshift=5cm,yshift=-7cm]
         \foreach \n[count=\x] / \ausgabe in {A/00,B/00,D/01,C/00,E/11}{
            \node[state] at (\x*360/5:4cm) (\n) {\n/\ausgabe};
         }
         \tikzset{every loop/.style={looseness=5}}
         \path[-{Latex}] (A) edge[loop above] node {11,00} ();
         \path[-{Latex}] (A) edge[bend right] node[above] {10} (B);
         \path[-{Latex}] (A) edge[bend right] node[left] {01} (C);
         \path[-{Latex}] (B) edge[loop left] node[above left] {00} ();
         \path[-{Latex}] (B) edge node[above] {11} (A);
         \path[-{Latex}] (B) edge[bend left,out=0] node[left] {01} (D);
         \path[-{Latex}] (B) edge[bend right=10] node[above right] {10} (C);
         \path[-{Latex}] (D) edge[loop below] node[below] {00} ();
         \path[-{Latex}] (D) edge[bend left] node[left] {10} (B);
         \path[-{Latex}] (D) edge[bend left,in=180] node[above] {01} (C);
         \path[-{Latex}] (C) edge[loop below] node[below] {00} ();
         \path[-{Latex}] (C) edge[bend left] node[above] {10} (D);
         \path[-{Latex}] (C) edge node[above left,pos=.8] {01} (E);
         \path[-{Latex}] (E) edge[loop right] node[right] {00} ();
         \path[-{Latex}] (E) edge[bend left] node[above left] {01} (C);
         \path[-{Latex}] (E) edge[bend right=5] node[pos=.8,above right] {10} (B);
         \path[-{Latex}] (E) edge[bend right] node[right] {11} (A);

         \path[-{Latex}] (D) edge[bend left,looseness=2,out=120,in=90] node[left] {11} (A);
         \path[-{Latex}] (C) edge[bend right,looseness=2,out=230,in=270] node[right] {11} (A);
      \end{scope}
   \end{tikzpicture}
\end{center}

Der Graph ist offensichtlich konsistent und ``vollständig'', da für jede Eingabe
genau ein Ausgang pro Knoten existiert.

\subex{}

Wir haben 5 Zustände, weshalb 3 Bit zur Kodierung nötig sind.

\begin{ctabular}{ll}
   \hline
   Zustand & Kodierung \\\hline
   A & 000 \\
   B & 001 \\
   C & 010 \\
   D & 011 \\
   E & 100 \\
   \hline
\end{ctabular}

Die Zustandsübergangstabelle ist damit
\addtolength{\tabcolsep}{-3pt}
\begin{ctabular}{cccc|cc|cc|ccc}
   \hline
   Zustand & $Z_2^n$ & $Z_1^n$ & $Z_0^n$ & $X_C$ & $X_E$ & $Y_R$ & $Y_G$ &
   $Z_2^{n+1}$ & $Z_1^{n+1}$ & $Z_0^{n+1}$\\
   \hline\hline
   A & 0 & 0 & 0 & 0 & 0 & 0 & 0 & 0 & 0 & 0 \\
   A & 0 & 0 & 0 & 0 & 1 & 0 & 0 & 0 & 1 & 0 \\
   A & 0 & 0 & 0 & 1 & 0 & 0 & 0 & 0 & 0 & 1 \\
   A & 0 & 0 & 0 & 1 & 1 & 0 & 0 & X & X & X \\\hline

   B & 0 & 0 & 1 & 0 & 0 & 0 & 0 & 0 & 0 & 1 \\
   B & 0 & 0 & 1 & 0 & 1 & 0 & 0 & 0 & 1 & 1 \\
   B & 0 & 0 & 1 & 1 & 0 & 0 & 0 & 0 & 1 & 0 \\
   B & 0 & 0 & 1 & 1 & 1 & 0 & 0 & X & X & X \\\hline

   C & 0 & 1 & 0 & 0 & 0 & 0 & 0 & 0 & 1 & 0 \\
   C & 0 & 1 & 0 & 0 & 1 & 0 & 0 & 1 & 0 & 0 \\
   C & 0 & 1 & 0 & 1 & 0 & 0 & 0 & 0 & 1 & 1 \\
   C & 0 & 1 & 0 & 1 & 1 & 0 & 0 & X & X & X \\\hline

   D & 0 & 1 & 1 & 0 & 0 & 0 & 1 & 0 & 1 & 1 \\
   D & 0 & 1 & 1 & 0 & 1 & 0 & 1 & 0 & 1 & 0 \\
   D & 0 & 1 & 1 & 1 & 0 & 0 & 1 & 0 & 0 & 1 \\
   D & 0 & 1 & 1 & 1 & 1 & 0 & 1 & X & X & X \\\hline

   E & 1 & 0 & 0 & 0 & 0 & 1 & 1 & 1 & 0 & 0 \\
   E & 1 & 0 & 0 & 0 & 1 & 1 & 1 & 0 & 1 & 0 \\
   E & 1 & 0 & 0 & 1 & 0 & 1 & 1 & 0 & 0 & 1 \\
   E & 1 & 0 & 0 & 1 & 1 & 1 & 1 & X & X & X \\\hline

     & 1 & 0 & 1 & X & X & X & X & X & X & X \\
     & 1 & 1 & 0 & X & X & X & X & X & X & X \\
     & 1 & 1 & 1 & X & X & X & X & X & X & X \\\hline
\end{ctabular}

\subex{Minimierung}
Unter Ausnutzung der Don't-Cares minimieren sich die Ausgänge zu
\begin{eqnarray*}
   Y_R & = & Z_2^n \\
   Y_G & = & Z_2^n + Z_1^nZ_0^n
\end{eqnarray*}

\subex{Zustandstabelle D-Flipflops}
Da für D-Flipflops gilt $Q^{n+1} = D^n$, sind die Ansteuersignale genau die
Bits, die den Folgezustand codieren.

\addtolength{\tabcolsep}{-3pt}
\begin{ctabular}{cccc|cc|cc|ccc>{\columncolor{lightgray!50}}c>{\columncolor{lightgray!50}}c>{\columncolor{lightgray!50}}c}
   \hline
   Zustand & $Z_2^n$ & $Z_1^n$ & $Z_0^n$ & $X_C$ & $X_E$ & $Y_R$ & $Y_G$ &
   $Z_2^{n+1}$ & $Z_1^{n+1}$ & $Z_0^{n+1}$ & $D_2^n$ & $D_1^n$ & $D_0^n$ \\
   \hline\hline
   A & 0 & 0 & 0 & 0 & 0 & 0 & 0 & 0 & 0 & 0 & 0 & 0 & 0 \\
   A & 0 & 0 & 0 & 0 & 1 & 0 & 0 & 0 & 1 & 0 & 0 & 1 & 0 \\
   A & 0 & 0 & 0 & 1 & 0 & 0 & 0 & 0 & 0 & 1 & 0 & 0 & 1 \\
   A & 0 & 0 & 0 & 1 & 1 & 0 & 0 & X & X & X & X & X & X \\\hline

   B & 0 & 0 & 1 & 0 & 0 & 0 & 0 & 0 & 0 & 1 & 0 & 0 & 1 \\
   B & 0 & 0 & 1 & 0 & 1 & 0 & 0 & 0 & 1 & 1 & 0 & 1 & 1 \\
   B & 0 & 0 & 1 & 1 & 0 & 0 & 0 & 0 & 1 & 0 & 0 & 1 & 0 \\
   B & 0 & 0 & 1 & 1 & 1 & 0 & 0 & X & X & X & X & X & X \\\hline

   C & 0 & 1 & 0 & 0 & 0 & 0 & 0 & 0 & 1 & 0 & 0 & 1 & 0 \\
   C & 0 & 1 & 0 & 0 & 1 & 0 & 0 & 1 & 0 & 0 & 1 & 0 & 0 \\
   C & 0 & 1 & 0 & 1 & 0 & 0 & 0 & 0 & 1 & 1 & 0 & 0 & 1 \\
   C & 0 & 1 & 0 & 1 & 1 & 0 & 0 & X & X & X & X & X & X \\\hline

   D & 0 & 1 & 1 & 0 & 0 & 0 & 1 & 0 & 1 & 1 & 0 & 1 & 1 \\
   D & 0 & 1 & 1 & 0 & 1 & 0 & 1 & 0 & 1 & 0 & 0 & 1 & 0 \\
   D & 0 & 1 & 1 & 1 & 0 & 0 & 1 & 0 & 0 & 1 & 0 & 0 & 1 \\
   D & 0 & 1 & 1 & 1 & 1 & 0 & 1 & X & X & X & X & X & X \\\hline

   E & 1 & 0 & 0 & 0 & 0 & 1 & 1 & 1 & 0 & 0 & 1 & 0 & 0 \\
   E & 1 & 0 & 0 & 0 & 1 & 1 & 1 & 0 & 1 & 0 & 0 & 1 & 0 \\
   E & 1 & 0 & 0 & 1 & 0 & 1 & 1 & 0 & 0 & 1 & 0 & 0 & 1 \\
   E & 1 & 0 & 0 & 1 & 1 & 1 & 1 & X & X & X & X & X & X \\\hline

     & 1 & 0 & 1 & X & X & X & X & X & X & X & X & X & X \\
     & 1 & 1 & 0 & X & X & X & X & X & X & X & X & X & X \\
     & 1 & 1 & 1 & X & X & X & X & X & X & X & X & X & X \\\hline
\end{ctabular}

\ex{Schaltwerksminimierung}{2 + 4 + 9 + 9 + 4 = 28}
\subex{Automatentyp}
Da die Ausgaben allein von den Zuständen abhängig sind (Notation in den Zustandsknoten), handelt es sich um einen MOORE-Automaten.

\subex{Zustandstabelle}
Wir verwenden X als Eingabe und Y als Ausgabe. 

\begin{ctabular}{c|c|ccc}
   \hline
   aktueller Zustand & aktuelle Ausgabe & \multicolumn{3}{c}{Folgezustand} \\
                     & Y                & X = & 0 & 1 \\ \hline
   A  & 0                &     & D & B \\
   B  & 0                &     & B & D \\
   C  & 0                &     & D & B \\
   D  & 1                &     & B & F \\
   E  & 1                &     & F & A \\
   F  & 1                &     & E & C \\
\end{ctabular}

\subex{Äquivalente Zustände: Tabelle}

\noindent\textbf{Gleiche Ausgabe}

Gruppen A-B-C und D-E-F

\noindent\bigskip\textbf{Folgezustände}

\begin{ctabular}{ll}
   Folgezustände Gruppe A-B-C & D-B-D (für X = 0) \\
                               & B-D-B (für X = 1) \\
   Folgezustände Gruppe D-E-F & B-F-E (für X = 0) \\
                               & F-A-C (für X = 1) \\
\end{ctabular}

\noindent\bigskip\textbf{Split}

Nach D-B-D bzw. B-D-B, aus denen D nicht in A-B-C vorkommt, muss A-B-C
aufgeteilt werden in A-C und B. Nach B-F-E muss D-E-F in D und E-F geteilt
werden.

Verbleibende Gruppen gleicher Ausgabe: A-C, B, D, E-F.

\noindent\bigskip\textbf{Folgezustände}

\begin{ctabular}{ll}
   Folgezustände Gruppe A-C   & D-D (für X = 0) \\
                               & B-B (für X = 1) \\
   Folgezustände Gruppe B     & B   (für X = 0) \\
                               & D   (für X = 1) \\
   Folgezustände Gruppe E-F   & F-E (für X = 0) \\
                               & A-C (für X = 1) \\
   Folgezustände Gruppe D     & B   (für X = 0) \\
                               & F   (für X = 1) \\
\end{ctabular}

\noindent\bigskip\textbf{Split}

Nach A-C (Folgezustand) muss E-F gesplittet werden. Da A-C (aktueller Zustand) erhalten bleibt, ergibt sich als Endergebnis:

\begin{ctabular}{c|c|ccc}
   \hline
   aktueller Zustand & aktuelle Ausgabe & \multicolumn{3}{c}{Folgezustand} \\
                     & Y                & X = & 0 & 1 \\ \hline
   A-C  & 0                &     & D & B \\
   B  & 0                &     & B & D \\
   D  & 1                &     & B & F \\
   E  & 1                &     & F & A-C \\
   F  & 1                &     & E & A-C \\
\end{ctabular}

\subex{Äquivalente Zustände: Implikationstafel}

\def\cell#1{\multicolumn{1}{|c|}{\begin{tabular}{c}#1\end{tabular}}}
   \newcommand{\cancel}[1]{%
      \tikz[baseline=(tocancel.base)]{
         \node[inner sep=0pt,outer sep=0pt] (tocancel) {#1};
         \draw[red,thick] (tocancel.south west) -- (tocancel.north east);
      }%
   }%
   \newcommand{\Xout}{\cell{\huge X}% placeholder in case I am motivated to figure out
      % how to draw a cross (see: 
      % http://tex.stackexchange.com/questions/156162/strike-out-a-table-cell )
   }

   \begin{center}
      \begin{tabular}{cccccc}
         \cline{2-2}
         B & \cell{D-B\\B-D} &                  &         &                 &                 \\\cline{2-3}
         C & \cell{D-D\\B-B} & \cell{B-D\\D-B}  &         &                 &                 \\\cline{2-4}
         D & \Xout           & \Xout            & \Xout   &                 &                 \\\cline{2-5}
         E & \Xout           & \Xout            & \Xout   & \cell{B-F\\F-A} &                 \\\cline{2-6}
         F & \Xout           & \Xout            & \Xout   & \cell{B-E\\F-C} & \cell{F-E\\A-C} \\\cline{2-6}
           & A               & B                & C       & D               & E \\
      \end{tabular}

      \begin{tabular}{cccccc}
         \cline{2-2}
         B & \cell{\cancel{D-B}\\\cancel{B-D}} &                  &         &                 &                 \\\cline{2-3}
         C & \cell{D-D\\B-B} & \cell{B-D\\D-B}  &         &                 &                 \\\cline{2-4}
         D & \Xout           & \Xout            & \Xout   &                 &                 \\\cline{2-5}
         E & \Xout           & \Xout            & \Xout   & \cell{B-F\\F-A} &                 \\\cline{2-6}
         F & \Xout           & \Xout            & \Xout   & \cell{B-E\\F-C} & \cell{F-E\\A-C} \\\cline{2-6}
           & A               & B                & C       & D               & E \\
      \end{tabular}

      \begin{tabular}{cccccc}
         \cline{2-2}
         B & \Xout &                  &         &                 &                 \\\cline{2-3}
         C & \cell{D-D\\B-B} & \cell{\cancel{B-D}\\\cancel{D-B}}  &         &                 &                 \\\cline{2-4}
         D & \Xout           & \Xout            & \Xout   &                 &                 \\\cline{2-5}
         E & \Xout           & \Xout            & \Xout   & \cell{B-F\\F-A} &                 \\\cline{2-6}
         F & \Xout           & \Xout            & \Xout   & \cell{B-E\\F-C} & \cell{F-E\\A-C} \\\cline{2-6}
           & A               & B                & C       & D               & E \\
      \end{tabular}

      \begin{tabular}{cccccc}
         \cline{2-2}
         B & \Xout &                  &         &                 &                 \\\cline{2-3}
         C & \cell{D-D\\B-B} & \Xout  &         &                 &                 \\\cline{2-4}
         D & \Xout           & \Xout            & \Xout   &                 &                 \\\cline{2-5}
         E & \Xout           & \Xout            & \Xout   & \cell{\cancel{B-F}\\\cancel{F-A}} &                 \\\cline{2-6}
         F & \Xout           & \Xout            & \Xout   & \cell{\cancel{B-E}\\\cancel{F-C}} & \cell{F-E\\A-C} \\\cline{2-6}
           & A               & B                & C       & D               & E \\
      \end{tabular}

      \begin{tabular}{cccccc}
         \cline{2-2}
         B & \Xout &                  &         &                 &                 \\\cline{2-3}
         C & \cell{D-D\\B-B} & \Xout  &         &                 &                 \\\cline{2-4}
         D & \Xout           & \Xout            & \Xout   &                 &                 \\\cline{2-5}
         E & \Xout           & \Xout            & \Xout   & \Xout &                 \\\cline{2-6}
         F & \Xout           & \Xout            & \Xout   & \Xout & \cell{\cancel{F-E}\\\cancel{A-C}} \\\cline{2-6}
           & A               & B                & C       & D               & E \\
      \end{tabular}

      \begin{tabular}{cccccc}
         \cline{2-2}
         B & \Xout &                  &         &                 &                 \\\cline{2-3}
         C & \cell{D-D\\B-B} & \Xout  &         &                 &                 \\\cline{2-4}
         D & \Xout           & \Xout            & \Xout   &                 &                 \\\cline{2-5}
         E & \Xout           & \Xout            & \Xout   & \Xout &                 \\\cline{2-6}
         F & \Xout           & \Xout            & \Xout   & \Xout & \Xout \\\cline{2-6}
           & A               & B                & C       & D               & E \\
      \end{tabular}
   \end{center}

   Da D$\equiv$D und B$\equiv$B ist, ist auch A$\equiv$C. Wir erhalten erneut

   \smallskip

   \begin{ctabular}{c|c|ccc}
      \hline
      aktueller Zustand & aktuelle Ausgabe & \multicolumn{3}{c}{Folgezustand} \\
                        & Y                & X = & 0 & 1 \\ \hline
      A-C  & 0                &     & D & B \\
      B  & 0                &     & B & D \\
      D  & 1                &     & B & F \\
      E  & 1                &     & F & A-C \\
      F  & 1                &     & E & A-C \\
   \end{ctabular}

   \subex{Minimierter Zustandsgraph}
   A-C wird nur als A bezeichnet!

   \begin{center}
      \begin{tikzpicture}[node distance=3cm]
         \node[state] (A)              {A/0};
         \node[state] (B) [right of=A] {B/0};
         \node[state] (D) [below of=B] {D/1};
         \node[state] (F) [below of=A] {E/1};
         \node[state] (E) [left of=F]  {F/1};

         \path[->] (A) edge             node[above right] {0} (D)
         (A) edge             node[above]       {1} (B)
         (B) edge[loop right] node[right]       {0} (B)
         (B) edge[bend right] node[right]       {1} (D)
         (D) edge[bend right] node[left]        {0} (B)
         (D) edge             node[below]       {1} (F)
         (E) edge[bend right] node[above]       {0} (F)
         (E) edge             node[above left]  {1} (A)
         (F) edge[bend right] node[below]       {0} (E)
         (F) edge             node[left]        {1} (A)
         ;
      \end{tikzpicture}
   \end{center}

   \end{document}
