\documentclass{CInf_practice}

\sheet{7}{Schaltwerke}

\begin{document}
\cinftitle

\ex{Schaltwerksentwurf}{4 + 16 + 8 + 2 + 6 = 36}

\subex{Ein- \& Ausgaben}
\noindent Eingaben:
\begin{enumerate}[align=left,leftmargin=\marginparwidth]
   \item[$X_C$] 50 Cent einwerfen oder nicht
   \item[$X_E$] Euromünze einwerfen oder nicht
\end{enumerate}
Ausgaben:
\begin{enumerate}[align=left,leftmargin=\marginparwidth]
   \item[$Y_R$] 50 Cent auswerfen oder nicht
   \item[$X_G$] Getränk auswerfen oder nicht
\end{enumerate}

\subex{Zustandsgraph}

Unser Automat ist insofern etwas kundenunfreundlich, als er immer in den
Ausgangszustand geht, sollte man es irgendwie hinbekommen, gleichzeitig ein
50-Cent-Münze und ein 1-Euro-Stück in den Schlitz zu quetschen. Aber wer den
Automaten so misshandelt, verdient ohnehin keine Limo.

\usetikzlibrary{automata,positioning,arrows.meta}
\begin{center}
   \hspace{-2cm}\begin{tikzpicture}[node distance=2cm,very thick,yscale=.7]
      \node[scale=.8,draw,rounded corners=5pt,fill=lightgray!50,text width=4cm] (legend) {
         \makebox[4cm]{Legende}\\
         \hrulefill \\
         An Zuständen: \makebox[4cm]{$Z/Y_R Y_G$} \\ An Übergängen: \makebox[4cm]{ $X_CX_E$ }
      };

      \tikzset{state/.append style={execute at begin node=$, execute at end node=$}}
      \begin{scope}[xshift=5cm,yshift=-7cm]
         \foreach \n[count=\x] in {A,B,D,C,E}{
            \node[state] at (\x*360/5:4cm) (\n) {\n/00};
         }
         \tikzset{every loop/.style={looseness=5}}
         \path[-{Latex}] (A) edge[loop above] node {11,00} ();
         \path[-{Latex}] (A) edge[bend right] node[above] {10} (B);
         \path[-{Latex}] (A) edge[bend right] node[left] {01} (C);
         \path[-{Latex}] (B) edge[loop left] node[above left] {00} ();
         \path[-{Latex}] (B) edge node[above] {11} (A);
         \path[-{Latex}] (B) edge[bend left,out=0] node[left] {01} (D);
         \path[-{Latex}] (B) edge[bend right=10] node[above right] {10} (C);
         \path[-{Latex}] (D) edge[loop below] node[below] {00} ();
         \path[-{Latex}] (D) edge[bend left] node[left] {10} (B);
         \path[-{Latex}] (D) edge[bend left,in=180] node[above] {01} (C);
         \path[-{Latex}] (C) edge[loop below] node[below] {00} ();
         \path[-{Latex}] (C) edge[bend left] node[above] {10} (D);
         \path[-{Latex}] (C) edge node[above left,pos=.8] {01} (E);
         \path[-{Latex}] (E) edge[loop right] node[right] {00} ();
         \path[-{Latex}] (E) edge[bend left] node[above left] {01} (C);
         \path[-{Latex}] (E) edge[bend right=5] node[pos=.8,above right] {10} (B);
         \path[-{Latex}] (E) edge[bend right] node[right] {11} (A);

         \path[-{Latex}] (D) edge[bend left,looseness=2,out=120,in=90] node[left] {11} (A);
         \path[-{Latex}] (C) edge[bend right,looseness=2,out=230,in=270] node[right] {11} (A);
      \end{scope}
   \end{tikzpicture}
\end{center}
\end{document}
