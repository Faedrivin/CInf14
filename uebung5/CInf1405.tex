\documentclass{CInf_practice}

\sheet{5}{Programmierbare Bausteine, Hazards und Flipflops}
\begin{document}
\cinftitle

\ex{Bit-Slice-ALU}{6 + 8 + 16 + 4 = 34}
\newlength{\vlinedist}
\newlength{\hlinedist}
\newlength{\hcablelength}
\newlength{\vcablelength}

\setlength{\vlinedist}{.3cm}
\setlength{\hlinedist}{.2cm}
\setlength{\hcablelength}{33\vlinedist}
\setlength{\vcablelength}{11\hlinedist}

\usetikzlibrary{intersections,calc,positioning}
\tikzstyle{inverter}=[font=\tiny,minimum height=.3cm,
   text width=.3cm,rectangle,anchor=center,
   align=center,
   inner sep=0,
draw ]
\tikzstyle{and gate}=[rectangle,font=\tiny,align=center,draw,fill=white,inner
sep=1pt,minimum height=.7em]
\begin{center}
   \begin{tikzpicture}
      \foreach \x in {1,...,10}
      {
         \draw[name path global/.expanded=h \x] (0,\hlinedist*\x) --
         ++(\hcablelength,0);
      }
      \foreach \x in {0,...,31} {
         \draw[name path global/.expanded=v \x] (\vlinedist*\x, \vcablelength)
         node[name=v \x start] {} -- ++(0,-\vcablelength-5*\hlinedist);
         \node[and gate] at ($(v \x start) + (0,-\vcablelength-\hlinedist)$) {\&};
      }
      \foreach \x / \y in {1/s_1,2/s_0,3/c_{i-1},4/a_i,5/b_i}
      {
         \node[anchor=east] (\y) at ($(0,\vcablelength+\hlinedist-2\hlinedist*\x) + (-1cm,0)$) {$\y$};
         \node[inverter,below right=\hlinedist-.15cm-.5\pgflinewidth and 15pt of \y.east] (inv-\y) {1};
         \draw (\y) -- ++(5cm,0);
         \node[draw,shape=circle,inner sep=.5pt,right=0pt-\pgflinewidth of inv-\y] (inv-\y-circle) {};
         \draw (inv-\y-circle) -- ++(1cm,0);
         \draw (inv-\y.west) -| ($(\y.east) + (10pt,0)$);
      }
      \draw (0,-2.5\hlinedist) -- ++(\hcablelength,0) node[and gate] {$\ge$} -- +(1em,0) ++(2em,0) node {$y_i$};
      \draw (0,-4\hlinedist) -- ++(\hcablelength,0) node[and gate] {$\ge$} -- +(1em,0) ++(2em,0) node {$c_i$};
      \node[rectangle,draw,very thick,below=2\hlinedist of b_i] {\textbf{PROM}};
      % \foreach \x [evaluate=\x as \i using int(\x+16)]in {0,...,15}{
      %    \fill[name intersections={of={h 9} and {v \x}}] (intersection-1) circle (1pt);
      %    \fill[name intersections={of={h 10} and {v \i}}] (intersection-1) circle (1pt);
      % }
      % \foreach \x in {0,...,31}{
      %    \ifthenelse{\x<8 \OR \(\x<24 \AND \x>15\)}{
      %       \fill[name intersections={of={h 7} and {v \x}}] (intersection-1) circle (1pt);
      %    }{}
      %    \ifthenelse{\x>7 \AND \x<16 \OR \x>23}{
      %       \fill[name intersections={of={h 8} and {v \x}}] (intersection-1) circle (1pt);
      %    }{}
      % }
      \newcounter{num}
      \foreach \line[count=\x,evaluate=\line as \l using int(\line-1),
         evaluate=\line as \t using int(\line+1),
      ] in {1,3,5,...,10}{
         %% ODD ROWS %%%%%%%%%%%%%%%%
         \pgfmathsetmacro{\numdots}{pow(2,\x-1)}
         \ifthenelse{\line=1}{
            \foreach \c in {0,2,...,30}{
               \fill[name intersections={of={h \line} and {v \c}}] (intersection-1) circle (1pt);
            }
         }{
            \setcounter{num}{0}
            \whiledo{\value{num}<32}{
               \foreach \n in {1,...,\numdots}{
                  \ifthenelse{\value{num}<32}{\fill[name intersections={of={h \line} and {v \arabic{num}}}] (intersection-1) circle (1pt);
                     \stepcounter{num}
                  }{}
               }
               \foreach \n in {1,...,\numdots}{\stepcounter{num}}
            }
         }
         %% END ODD ROWS %%%%%%%%%%%%

         %% EVEN ROWS %%%%%%%%%%%%%%%%
         \ifthenelse{\line=1}{
            \foreach \c in {0,2,...,30}{
               \pgfmathsetmacro{\index}{int(pow(2,\x-1)+\c)}
               \fill[name intersections={of={h \t} and {v \index}}] (intersection-1) circle (1pt);
            }
         }{
            \setcounter{num}{0}
            \whiledo{\value{num}<32}{
               \foreach \n in {1,...,\numdots}{
                  \pgfmathsetmacro{\index}{int(pow(2,\x-1)+\value{num})}
                  \ifthenelse{\index<32 \AND \value{num}<32}{\fill[name
                     intersections={of={h \t} and {v \index}}] (intersection-1) circle (1pt);
                     \stepcounter{num}
                  }{}
               }
               \foreach \n in {1,...,\numdots}{\stepcounter{num}}
            }
         }
         %% END EVEN ROWS %%%%%%%%%%%%
      }
   \end{tikzpicture}
\end{center}

\setlength{\hcablelength}{17\vlinedist}
\begin{center}
   \begin{tikzpicture}
      \foreach \x in {1,...,10}
      {
         \draw[name path global/.expanded=h \x] (0,\hlinedist*\x) --
         ++(\hcablelength,0);
      }
      \foreach \x in {0,...,15} {
         \draw[name path global/.expanded=v \x] (\vlinedist*\x, \vcablelength)
         node[name=v \x start] {} -- ++(0,-\vcablelength-5*\hlinedist);
         \node[and gate] at ($(v \x start) + (0,-\vcablelength-\hlinedist)$) {\&};
      }
      \foreach \x / \y in {1/s_1,2/s_0,3/c_{i-1},4/a_i,5/b_i}
      {
         \node[anchor=east] (\y) at ($(0,\vcablelength+\hlinedist-2\hlinedist*\x) + (-1cm,0)$) {$\y$};
         \node[inverter,below right=\hlinedist-.15cm-.5\pgflinewidth and 15pt of \y.east] (inv-\y) {1};
         \draw (\y) -- ++(5cm,0);
         \node[draw,shape=circle,inner sep=.5pt,right=0pt-\pgflinewidth of inv-\y] (inv-\y-circle) {};
         \draw (inv-\y-circle) -- ++(1cm,0);
         \draw (inv-\y.west) -| ($(\y.east) + (10pt,0)$);
      }
      \draw (0,-2.5\hlinedist) -- ++(\hcablelength,0) node[and gate] {$\ge$} -- +(1em,0) ++(2em,0) node {$y_i$};
      \draw (0,-4\hlinedist) -- ++(\hcablelength,0) node[and gate] {$\ge$} -- +(1em,0) ++(2em,0) node {$c_i$};
      \node[rectangle,draw,very thick,below=2\hlinedist of b_i] {\textbf{PAL}};

   \end{tikzpicture}
\end{center}
\end{document}
