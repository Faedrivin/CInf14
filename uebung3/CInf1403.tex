\documentclass{CInf_practice}

\sheet{3}{Schaltfunktionen und Minimierung}

\begin{document}

\cinftitle

\ex{Normalformen}{6 + 6 + 6 + 6 + 6 = 30}
Anmerkung: Außer Duplikation von Termen wurde fast immer das 5. Axiom (vorwärts
oder rückwärts) zur Vereinfachung angewandt.

\begin{enumerate}[label=\alph{*})]
   \item
      \begin{equation*}
         f_{DKF} = \comp a\comp b \comp c + \comp a b\comp c + \comp a b c + 
         a b\comp c \\
      \end{equation*}
   \item
      \begin{eqnarray*}
         f_{DNF} & = & \comp a\comp b \comp c + \comp a b\comp c + \comp a b c + 
         a b\comp c \\
                 & = & \comp a \comp b \comp c + \comp a b \comp c + \comp a b
                       \comp c + \comp a b \comp c + \comp a b c + a b \comp c \\
                 & = & \comp a \comp c + b \comp c + \comp a b
      \end{eqnarray*}
   \item
      \begin{equation*}
         f_{KKN} = (a + b + \comp c) \cdot (\comp a + b + c) \cdot (\comp a
         + b  +\comp c) \cdot (\comp a + \comp b  +\comp c)
      \end{equation*}
   \item
      \begin{eqnarray*}
         f_{KKN} & = & (a + b + \comp c) \cdot (\comp a + b + c) \cdot (\comp a + b  +\comp c) \cdot (\comp a + \comp b  +\comp c) \\
                 & = & (a + b + \comp c) \cdot (\comp a + b + c) \cdot (\comp a
                        + b  +\comp c) \cdot (\comp a + \comp b  +\comp c) \cdot (\comp a + b + \comp c)\\
                 & = & (a + b + \comp c) \cdot ((\comp a + b) + (c\comp e)) \cdot ((\comp a + \comp c) + (b\comp b)) \\
                 & = & (a + b \comp c) \cdot (\comp a + b) \cdot (\comp a + \comp c) \\
                 & = & (\comp a + b) \cdot (\comp c + ((a + b)\cdot\comp a)) \\
                 & = & (\comp a + b) \cdot (\comp c + (a\comp a + \comp a b)) \\
                 & = & (\comp a + b) \cdot (\comp c + \comp a b) \\
                 & = & (\comp a + b) \cdot (\comp a + \comp c) \cdot (b + \comp c)
      \end{eqnarray*}

   \item \hspace{\linewidth} % weirdly, i must put space here in order for the
                             % tikzpicture to start at the next line

   \begin{tikzpicture}[node distance=.5cm] % TODO: Encapsulate some of this stuff in macros
      \node (a) {$a$};
      \node[right=of a] (b) {$b$};
      \node[right=of b] (c) {$c$};
      \draw[name path=a-line](a) -- ++(0,-7cm);
      \draw[name path=b-line](b) -- ++(0,-7cm);
      \draw[name path=c-line](c) -- ++(0,-7cm);

      \node[and port, below right=.5cm and 2 cm of c] (and-1) {};
      \draw[name path=a-and-1] (a) |- (and-1.145);
      \draw[name path=b-and-1] (b) |- (and-1.180);
      \draw[name path=c-and-1] (c) |- (and-1.215);
      \fill[black,name intersections={of=a-line and a-and-1}] (intersection-1) circle (2pt);
      \fill[black,name intersections={of=b-line and b-and-1}] (intersection-1) circle (2pt);
      \fill[black,name intersections={of=c-line and c-and-1}] (intersection-1) circle (2pt);
      \begin{scope} % scope to only clip circles
      \clip (and-1.north west) rectangle ++(-3cm,-3cm); % to cut circles
      \filldraw[fill=white] ($(and-1.145)$) circle (2pt);
      \filldraw[fill=white] ($(and-1.180)$) circle (2pt);
      \filldraw[fill=white] ($(and-1.215)$) circle (2pt);
      \end{scope}

      % second and
      \node[and port, below=of and-1] (and-2) {};
      \draw[name path=a-and-2] (a) |- (and-2.145);
      \draw[name path=b-and-2] (b) |- (and-2.180);
      \draw[name path=c-and-2] (c) |- (and-2.215);
      \fill[black,name intersections={of=a-line and a-and-2}] (intersection-1) circle (2pt);
      \fill[black,name intersections={of=b-line and b-and-2}] (intersection-1) circle (2pt);
      \fill[black,name intersections={of=c-line and c-and-2}] (intersection-1) circle (2pt);
      \begin{scope} % scope to only clip circles
      \clip (and-2.north west) rectangle ++(-3cm,-3cm); % to cut circles
      \filldraw[fill=white] ($(and-2.145)$) circle (2pt);
      \filldraw[fill=white] ($(and-2.215)$) circle (2pt);
   \end{scope}

      % third and
      \node[and port, below=of and-2] (and-3) {};
      \draw[name path=a-and-3] (a) |- (and-3.145);
      \draw[name path=b-and-3] (b) |- (and-3.180);
      \draw[name path=c-and-3] (c) |- (and-3.215);
      \fill[black,name intersections={of=a-line and a-and-3}] (intersection-1) circle (2pt);
      \fill[black,name intersections={of=b-line and b-and-3}] (intersection-1) circle (2pt);
      \fill[black,name intersections={of=c-line and c-and-3}] (intersection-1) circle (2pt);
      \begin{scope} % scope to only clip circles
      \clip (and-3.north west) rectangle ++(-3cm,-3cm); % to cut circles
      \filldraw[fill=white] ($(and-3.145)$) circle (2pt);
   \end{scope}

      % fourth and
      \node[and port, below=of and-3] (and-4) {};
      \draw[name path=a-and-4] (a) |- (and-4.145);
      \draw[name path=b-and-4] (b) |- (and-4.180);
      \draw[name path=c-and-4] (c) |- (and-4.215);
      \fill[black,name intersections={of=a-line and a-and-4}] (intersection-1) circle (2pt);
      \fill[black,name intersections={of=b-line and b-and-4}] (intersection-1) circle (2pt);
      \fill[black,name intersections={of=c-line and c-and-4}] (intersection-1) circle (2pt);
      \begin{scope} % scope to only clip circles
      \clip (and-4.north west) rectangle ++(-3cm,-3cm); % to cut circles
      \filldraw[fill=white] ($(and-4.215)$) circle (2pt);
   \end{scope}

   % or
   
   \node[or port] (or-1) at ($(and-2)!0.5!(and-3)+(4cm,0)$) {};
   \draw (and-1.east) -- ++(2cm,0) |- (or-1.145);
   \draw (and-2.east) -- ++(1cm,0) |- (or-1.162.5);
   \draw (and-3.east) -- ++(1cm,0) |- (or-1.197.5);
   \draw (and-4.east) -- ++(2cm,0) |- (or-1.215);

   \draw (or-1.east) -- ++(2cm,0) ++(5pt,0) node () {$f$};
   \end{tikzpicture}
   \end{enumerate}
   \ex{Schaltungsanalyse}{6 + 6 + 3 + 3 = 18}
   \ex{KV-Minimierung}{6 + 8 + 8 = 22}
   \ex{KV-Minimierung mit 5 Variablen}{12}
   \ex{7-Segment-Anzeige}{2 + 8 + 8 = 18}

   \end{document}
