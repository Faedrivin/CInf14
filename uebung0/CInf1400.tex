\documentclass{CInf_practice}

\sheet{0}{Dualzahlen und Codes}

\begin{document}

\cinftitle



\ex{Binär-, Oktal- und Hexadezimalzahlen}{25}

\begin{center}
   \begin{tabular}{HChChChC}
      \rowstyle{\normalfont} 
      Dezimal & Binär & Oktal & Hexadezimal \\ \hline
      \temph 141.28125 & 10001101.01001 & 215.22 & 8D.48 \\ \hline
      181.625 & \temph 10110101.101 & 265.5 & B5.A \\ \hline
      132.625 & 10000100.101 & \temph 204.5 & 84.A \\ \hline
      315.75 & 100111011.11 & 473.6 & \temph 13B.C \\ \hline
   \end{tabular}
\end{center}



\ex{Binäre Addition \& Subtraktion}{10}

\begin{multicols}{2}
  \subex
  \begin{center}
    \begin{tabular}{Rr}
        1101 &   13 \\
      + 1110 & + 14 \\ \hline
      \rowcolor{green!10} 11011 & 27
    \end{tabular}
  \end{center}

  \subex
  \begin{center}
    \begin{tabular}{Rr}
         11001 &  25 \\
       - 10011 & -19 \\ \hline
    \end{tabular}$\Rightarrow$
    \begin{tabular}{Rr}
         11001 &  25 \\
       + 01101 & -19 \\ \hline
       \rowcolor{green!10} 00110 & 6
    \end{tabular}
  \end{center}
\end{multicols}



\ex{Binäre Multiplikation}{15}

\begin{center}
  \begin{tabular}{R}
    1100101\\
    $\cdot$0011011\\\hline
    1100101\\
    +1100101\hphantom{0}\\
    +0000000\hphantom{00}\\
    +1100101\hphantom{000}\\
    +1100101\hphantom{0000}\\ \hline
    \rowcolor{green!10}101010100111
  \end{tabular}
\end{center}



\ex{Binäre Division mit Zweierkomplement}{20}

\begin{center}
  \begin{tabular}{RR}
     & 01101\\
     \cline{2-2}101010 & \multicolumn{1}{|R}{1000100010} \\
    & -101010\hphantom{000}\\
    & +010110\hphantom{000}\\\hline
    & 11010\hphantom{000}\\
    & 110100\hphantom{00}\\
    & -101010\hphantom{00}\\
    & +010110\hphantom{00}\\\hline
    & $\not$1001010\hphantom{00}\\
    & 10101\hphantom{0}\\
    & 101010\\
    & -101010\\\hline
    & 0
   \end{tabular}
\end{center}



\ex{BCD-Addition}{20}

BCD steht für binary-coded decimal. Hierbei werden alle Ziffern durch ihre 4-
bit Binärdarstellung repräsentiert, d.h. \texttt{0} wird zu \texttt{0000}, 
\texttt{3} zu \texttt{0011}, etc.

\begin{multicols}{2}
  \begin{center}
    \begin{tabular}{RRR|R}
       5 &    9 &    1 & 591 \\
    0101 & 1001 & 0001 & 010110010001
    \end{tabular}
  \end{center}

  \begin{center}
    \begin{tabular}{RRR|R}
       4 &    3 &    7 & 437 \\
    0100 & 0011 & 0111 & 010000110111
    \end{tabular}
  \end{center}
\end{multicols}

Zur Addition werden zuerst die beiden BCDs als Binärzahlen addiert und wieder 
in BCD, also 4-bit Werte aufgeteilt.

\begin{center}
\begin{tabular}{R}
     010110010001 \\
   + 010000110111 \\ \hline
   \rowcolor{green!10} 100111001000
\end{tabular}
\end{center}

\texttt{100111001000} wird zu \texttt{1001 1100 1000}. Jeder dieser Werte wird 
auf $>$\texttt{1001} getestet, von rechts beginnend. Ist er größer, so wird 
\texttt{0110} hinzugerechnet (\texttt{6}), da nur 10 von 16 Bit-
Kombinationen zugelassen sind, und durch die Addition von 6 ein Überlaufbit 
erzeugt wird.

\begin{center}
\begin{tabular}{RlRR}
     1000 > 1001? & Nein.                           &  1000 & \temph 1000 \\
     1100 > 1001? & Ja. Addiere \texttt{0110}.      & 10010 & \temph 0010 \\
                  & Erhöhe \texttt{1001} um {\tt1}. &  1010 & \\
     1010 > 1001? & Ja. Addiere \texttt{0110}.      & 10000 & \temph 0000 \\
                  & Baue neuen Wert \texttt{0001}.  &  0001 & \temph 0001
\end{tabular}
\end{center}

Die Blöcke werden umgerechnet und das Ergebnis \texttt{0001000000101000} steht.
Zuletzt werden die Zahlen dekodiert:

\begin{center}
  \begin{tabular}{RRRR|R}
  0001 & 0000 & 0010 & 1000 & 0001000000101000\\
     1 &    0 &    2 &    8 & 1028 
  \end{tabular}
\end{center}



\ex{ASCII-Code}{10}

\begin{center}
  \begin{tabular}{B|CCCCC}
    Binär   & 1000111 & 1000101 & 1010011 & 1000011 & 1001000 \\ 
    Dezimal &      71 &      69 &      83 &      67 &      72 \\ \hline
            &       G &       E &       S &       C &       H 
  \end{tabular}
\end{center}
  
\begin{center}
  \begin{tabular}{B|CCCCC}
    Binär   & 1000001 & 1000110 & 1000110 & 1010100 & 0100001 \\
    Dezimal &      65 &      70 &      70 &      84 &      33 \\ \hline
            &       A &       F &       F &       T &       !
  \end{tabular}
\end{center}



\addex{Codeumwandlung}
Um von BCD in ASCII umzuwandeln muss lediglich jeder BCD Ziffer auf acht Bit erweitert und zur entstehenden Binärzahl der ASCII Code für 0 ($48_{10}$, also $00110000_2$) hinzuaddiert werden.

Beispiel: 23.

\begin{center}
  \begin{tabular}{B|RR}
    Dezimal & 2 & 3 \\
    BCD     & 0010 & 0011 \\
    8 bit BCD & 00000010 & 00000011 \\
    ASCII Code & 00110010 & 00110011 \\
    Dezimalwert & 50 & 51 \\
    Zeichen & 2 & 3
  \end{tabular}
\end{center}



\end{document}
