\documentclass{CInf_practice}

\sheet{10}{Beispiel CPUs HAM und H6809}

\begin{document}
\cinftitle

\ex{Befehlserweiterung der HAM}{5 + 7 + 9 + 5 + 3 + 3}

\subex{\bf SETA}
\lstinputlisting[language=RTeasy,firstline=50,lastline=53]{HAM_extended.rt}
\subex{\bf ADDI}
\lstinputlisting[language=RTeasy,firstline=55,lastline=58]{HAM_extended.rt}
\subex{\bf LOADI}
\lstinputlisting[language=RTeasy,firstline=60,lastline=64]{HAM_extended.rt}
\subex{\bf STOREI}
\lstinputlisting[language=RTeasy,firstline=66,lastline=70]{HAM_extended.rt}

\subex{Testprogramm}

Nach Ausführen des Tests muss an Speicherstelle 9 die Zahl 1 stehen und an 10
nur Einsen. Die Tabelle beinhaltet in der dritten Spalte eine zahl für den
Operanden in den unteren 8 Bit, die dem Opcode folgen (für Ein-Wort-Befehle) und
nach dem Komma den Operanden, der ggf. im zweiten Wort folgt.

\begin{assemblertable}
   0 & 9 & - & & SETA & - & Set accu to all ones \\\hline
   1 & A & -,2 & & ADDI & 2 & Add 2 to accu (will be 1 after) \\\hline
   3 & C & 9,- & & STOREI & 9 & Store accu at address positioned at mem(9)
   (which happens to be 9 again) \\\hline
   4 & B & A,- & & LOADI & 10 & Load value from address stored in mem(10) (which
   is again 10) \\\hline
   5 & 9 & - & & SETA & - & Set accu to all ones \\\hline
   6 & C & A,- & & STOREI & 10 & Store accu at address positioned at mem(10) \\\hline
   \vdots & & & & & & \\\hline
   9 & & & & .DB & 9 & Store value of 9 at position 9\\\hline
   A & & & & .DB & 10 & Store value of 10 at position 10\\\hline
\end{assemblertable}

\subex{Assemblat}

\lstinputlisting{aufg_1_test.mem} % sadly MikTeX can't interpret files without extensions which are not .tex here
                                  % wat ⦾ _ ⦾

\subex{Taktzahl}

Wenn man den FETCH-Zyklus ignoriert, der für alle Befehle gleich lang dauert,
ergeben sich folgende Laufzeiten:
\begin{ctabular}{>{\bf}lr}
   \toprule
   \normalfont{Befehl} & Taktzahl \\\midrule
   SETA & 4 \\
   ADDI & 4 \\
   LOADI & 5 \\
   STOREI & 5
\end{ctabular}

\ex{Collatz-Alg. auf der HAM}{18 + 10 + 4}

\subex{Assemblerprogramm}
\begin{assemblertable}
  00 & 1 & 13 & LOOP:   & LOAD   & VALUE  & load data VALUE     \\\hline
  01 & 4 & 14 &         & AND    & ODDBIT & check if even/odd   \\\hline
  02 & 6 & 0B &         & JUMPZ  & EVEN   & if even: jump       \\\hline
  03 & 1 & 13 & CHECK1: & LOAD   & VALUE  & else: check if 1    \\\hline
  04 & 4 & 15 &         & AND    & INVODD & if 1, then now 0    \\\hline
  05 & 6 & 12 &         & JUMPZ  & END    & if 0, then finish   \\\hline
  06 & 1 & 13 & ODD:    & LOAD   & VALUE  & odd case:           \\\hline
  07 & 3 & 13 &         & ADD    & VALUE  & multiply by 3 with  \\\hline
  08 & 3 & 13 &         & ADD    & VALUE  & \ additions and     \\\hline
  09 & 3 & 14 &         & ADD    & ODDBIT & add 1               \\\hline
  0A & 5 & 0D &         & JUMP   & COUNT  & count the step      \\\hline
  0B & 1 & 13 & EVEN:   & LOAD   & VALUE  &                     \\\hline
  0C & 8 & 00 &         & RSHIFT &        & divide by 2         \\\hline
  0D & 2 & 13 & COUNT:  & STORE  & VALUE  & store the new VALUE \\\hline
  0E & 1 & 16 &         & LOAD   & ERG    & load ERG            \\\hline
  0F & 3 & 14 &         & ADD    & ODDBIT & increment ERG       \\\hline
  10 & 2 & 16 &         & STORE  & ERG    & store ERG           \\\hline
  11 & 5 & 00 &         & JUMP   & LOOP   & repeat from start   \\\hline
  12 & 5 & 12 & END:    & JUMP   & END    & termination loop    \\\hline
  13 & 0 & 0D & VALUE   &        &      D & Input Value (13)    \\\hline
  14 & 0 & 01 & ODDBIT  &        &      1 & Mask: 000000000001  \\\hline
  15 & F & FE & INVODD  &        &    FFE & Mask: 111111111110  \\\hline
  16 & 0 & 00 & ERG     &        &      0 & Output Value        \\\hline
\end{assemblertable}

\subex{Handassemblierung}
\lstinputlisting{collatz_ham.mem}

\subex{Speicherbedarf und Laufzeit}
Speicherbedarf: $17 [Speicherstellen] \cdot 3 [\frac{Nibbles}{Speicherstelle}] \cdot 4 [\frac{Bit}{Nibble}] = 204 [Bit]$.

Laufzeit für $13$: 545 RT Takte bis \texttt{JUMPZ END} ausgeführt ist, danach beliebig viele Takte in \texttt{END: JUMP END}. 


\ex{Collatz-Alg. auf der H6809}{2 + 10 + 8 + 10 + 6}

\begin{assemblertable}
   & &  & INIT:     & LDX   & \#0  & Count iterations in X     \\\hline
   & &  & LOOP:     & LDA   & VALUE  & Load argument     \\\hline
   & &  &           & CMPA   & \#1  & If 1, abort     \\\hline
   & &  &           & BEQ   & END & \\\hline
   & &  &           & LDA   & VALUE & \textcolor{red}{necessary?} Load argument to subroutine \\\hline
   & &  &           & JSR  & COLLATZ & Compute next value \\\hline
   & &  &           & STA  & VALUE & Save new value \\\hline
   & &  &           & ADDX  & \#1 & Increment count \\\hline
   & &  &           & BRA  & LOOP & And so on  \\\hline
   & &  & COLLATZ:  & ANDA   & \#1  & check if odd     \\\hline
   & &  &           & BNE   & EVEN  & if not, branch     \\\hline
   & &  &           & LDA   & VALUE  & Restore VALUE     \\\hline
   & &  & ODD:      & ADDA   & VALUE  &  $\times 2$     \\\hline
   & &  &           & ADDA   & VALUE  &  $\times 3$     \\\hline
   & &  &           & ADDA   & \#1  &  $+1$     \\\hline
   & &  &           & RTS   &   &  Done.     \\\hline
   & &  & EVEN:     & RORA   &        & Divide by 2     \\\hline
   & &  &           & ANDA   & \#128   & Kill MSB     \\\hline
   & &  &           & RTS   &         & Done.     \\\hline
   & &  & END:      & STX   & ERG-1     & Store count.     \\\hline
   & &  &           & BRA   & 0         & Infinite loop.     \\\hline
   & &  &           & .BYTE   & 1     & Must be free to have low-byte of X stored at ERG    \\\hline
   & &  & ERG       & .BYTE   & 1     & Allocate 1 byte for ERG.     \\\hline
   & &  & VALUE     & .BYTE   & 1     & Allocate 1 byte for inpu.     \\\hline
\end{assemblertable}
\end{document}
