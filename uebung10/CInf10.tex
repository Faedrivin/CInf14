\documentclass{CInf_practice}

\sheet{10}{Beispiel CPUs HAM und H6809}

\begin{document}
\cinftitle

\ex{Befehlserweiterung der HAM}{5 + 7 + 9 + 5 + 3 + 3}

\subex{\bf SETA}
\lstinputlisting[language=RTeasy,firstline=50,lastline=53]{HAM_extended.rt}
\subex{\bf ADDI}
\lstinputlisting[language=RTeasy,firstline=55,lastline=58]{HAM_extended.rt}
\subex{\bf LOADI}
\lstinputlisting[language=RTeasy,firstline=60,lastline=64]{HAM_extended.rt}
\subex{\bf STOREI}
\lstinputlisting[language=RTeasy,firstline=66,lastline=70]{HAM_extended.rt}

\subex{Testprogramm}

Nach Ausführen des Tests muss an Speicherstelle 9 die Zahl 1 stehen und an 10
nur Einsen.

\begin{assemblertable}
   0 & 9 & - & & SETA & - & Set accu to all ones \\\hline
   1 & A & -,2 & & ADDI & 2 & Add 2 to accu (will be 1 after) \\\hline
   3 & C & 9,- & & STOREI & 9 & Store accu at address positioned at mem(9)
   (which happens to be 9 again) \\\hline
   4 & B & A,- & & LOADI & 10 & Load value from address stored in mem(10) (which
   is again 10) \\\hline
   5 & 9 & - & & SETA & - & Set accu to all ones \\\hline
   6 & C & A,- & & STOREI & 10 & Store accu at address positioned at mem(10) \\\hline
   \vdots & & & & & & \\\hline
   9 & & & & .DB & 9 & Store value of 9 at position 9\\\hline
   A & & & & .DB & 10 & Store value of 10 at position 10\\\hline
\end{assemblertable}

\subex{Assemblat}

\lstinputlisting{aufg_1_test}

\subex{Taktzahl}
\ex{Collatz-Alg. auf der HAM}{18 + 10 + 4}
\ex{Collatz-Alg. auf der H6809}{2 + 10 + 8 + 10 + 6}
\end{document}
