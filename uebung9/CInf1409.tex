\documentclass{CInf_practice}

\sheet{9}{RT-Optimierung und Realisierungsprinzipien für Steuerwerke}
\usetikzlibrary{automata,positioning,arrows.meta,fit,shapes.misc}
\lstset{xleftmargin=0pt,xrightmargin=0pt,morekeywords={fi,declare,register,array,bus,memory,goto,then,read}}

\begin{document}
\cinftitle

\ex{Optimierung eines RT-Programms}{10 + 7 + 5 + 8 = 30}

\subex{Optimierung}
Das Originalprogramm benötigt für 8 Fibonacci-Zahlen 44 Takte, das optimierte
nur 26.
\lstinputlisting{aufg1_optimised.rt}
\ex{One-Hot-Design und Zählersteuerung}{14 + 3 + 14 + 3 = 34}
\ex{Mikroprogrammiertes Steuerwerk}{7 + 2 + 4 + 8 = 21}
\ex{Mikrobefehlsformate}{5 + 5 + 5 = 15}
\end{document}
