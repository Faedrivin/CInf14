\documentclass{CInf_practice}

\sheet{4}{Quine-McCluskey und Koppelterme}


\usepackage{multicol}
\usetikzlibrary{matrix}
\tikzstyle{highlight}=[densely dotted,line width=.7pt,draw,rounded corners=3pt]
\tikzstyle{matrix node}=[anchor=center,draw,shape=rectangle,text width=1.5em,minimum height=1.5em,text centered,font=\scriptsize]
\tikzstyle{kv map}=[nodes={matrix node}, matrix of nodes, inner sep=0pt, outer sep=0pt, row sep=-.5pt, column sep=-.5pt]

\newcommand{\kvmeta}{
\draw[|-|] ($(m-2-1.north west) + (-.2em,0)$) -- node[left] () {$\scriptstyle x_0$} ++(0,-1.5em);
\draw[|-|] ($(m-1-2.north west) + (0,.2em)$) -- node[above] () {$\scriptstyle x_1$} ++(3em,0);
\draw[|-|] ($(m-1-3.north west) + (0,1.4em)$) -- node[above] () {$\scriptstyle x_2$} ++(3em,0);
\clip (m.north west) rectangle (m.south east);
}

\newcommand{\kvgroup}[5][]{\kvgrp[#1]{#2-#3}{#4-#5}}
\newcommand{\kvgrp}[3][]{\draw[highlight,#1] ($(m-#2.north west) + (2pt,-2pt)$) rectangle ($(m-#3.south east) + (-2pt,2pt)$);}
\newcommand{\kvgrpclp}[5][]{
\draw[highlight,#1] ($(m-#2.north west) + (-2pt,-2pt)$) rectangle ($(m-#3.south east) + (-2pt,2pt)$);
\draw[highlight,#1] ($(m-#4.north west) + (2pt,-2pt)$) rectangle ($(m-#5.south east) + (2pt,2pt)$);
}
%\def\kvgroup#1#2#3#4#5{\kvgrp{#1}{#2-#3}{#4-#5}}

\begin{document}
	

\cinftitle

\ex{Minimierung nach Quine-McCluskey}{20 + 10 = 30}
\subex{Minimierung}

\begin{align*}
f(a,b,c,d) = \sum\limits^{15}_{i=0} \text{\bf Min}^{4}_i \cdot r_i, &\text{ mit } r_i=1 \text{ für } i = \left\{0,2,4,7,8,10,11,15\right\}\\
& \text{ und mit Don't Care für } i=\left\{1,9\right\}
\end{align*}

%\begin{center}
%\begin{tabular}{CCCC|R}
%a & b & c & d & f \\ \hline
%0 & 0 & 0 & 0 & 1 \\
%0 & 0 & 0 & 1 & X \\
%0 & 0 & 1 & 0 & 1 \\
%0 & 0 & 1 & 1 & 0 \\
%0 & 1 & 0 & 0 & 1 \\
%0 & 1 & 0 & 1 & 0 \\
%0 & 1 & 1 & 0 & 0 \\
%0 & 1 & 1 & 1 & 1 \\
%1 & 0 & 0 & 0 & 1 \\
%1 & 0 & 0 & 1 & X \\
%1 & 0 & 1 & 0 & 1 \\
%1 & 0 & 1 & 1 & 1 \\
%1 & 1 & 0 & 0 & 0 \\
%1 & 1 & 0 & 1 & 0 \\
%1 & 1 & 1 & 0 & 0 \\
%1 & 1 & 1 & 1 & 1 \\
%\end{tabular}
%\end{center}

\textbf{1. disjunktive kanonische Normalform}

$f = \comp a \comp b \comp c \comp d
   + \comp a \comp b \comp c       d
   + \comp a \comp b       c \comp d 
   + \comp a       b \comp c \comp d
   + \comp a       b       c       d 
   +       a \comp b \comp c \comp d
   +       a \comp b \comp c       d   
   +       a \comp b       c \comp d 
   +       a \comp b       c       d 
   +       a       b       c       d$

\bigskip

\textbf{2. Primimplikanten}

\begin{center}
\begin{tabular}{ c | c >{$}c<{$} | c >{$}c<{$} | c >{$}c<{$}}
Klasse & \# & \text{Minterme} & \# & \text{Verschm. MinT} & \# & \text{Verschm. MinT}\\\hline
\multirow{1}{*}{$K_0$} & 0  & \comp a \comp b \comp c \comp d & 0,1   & \comp a \comp b \comp c         & 0,1-8,9  & \comp b \comp c \\
                       &    &                                 & 0,2   & \comp a \comp b         \comp d &&\\
                       &    &                                 & 0,4   & \comp a         \comp c \comp d &&\\
                       &    &                                 & 0,8   &         \comp b \comp c \comp d & 0,8-1,9  & \comp b \comp c\\
                       &    &                                 &       &                                 & 0,8-2,10 & \comp b \comp d\\
                      
\multirow{1}{*}{$K_1$} & 1  & \comp a \comp b \comp c       d & 1,9   &         \comp b \comp c       d &&\\
                       & 2  & \comp a \comp b       c \comp d & 2,10  &         \comp b       c \comp d &&\\
                       & 4  & \comp a       b \comp c \comp d &       & &&\\
                       & 8  &       a \comp b \comp c \comp d & 8,9   &       a \comp b \comp c         &&\\
                       &    &                                 & 8,10  &       a \comp b               d &&\\
                      
\multirow{1}{*}{$K_2$} & 9  &       a \comp b \comp c       d & 9,11  &       a \comp b               d &&\\
                       & 10 &       a \comp b       c \comp d & 10,11 &       a \comp b               d &&\\
                      
\multirow{1}{*}{$K_3$} & 7  & \comp a       b       c       d & 7,15  &               b       c       d &&\\
                       & 11 &       a \comp b       c       d & 11,15 &       a               c       d &&\\
                      
\multirow{1}{*}{$K_4$} & 15 &       a       b       c       d &       & &&\\
\end{tabular}
\end{center}

\textbf{3. wesentliche Primimplikanten}

\begin{center}
\begin{tabular}{>{$}l<{$}|cccccccc|c}
\text{Primimpl.}        & 0 & 2 & 4 & 7 & 8 &10 &11 &15 & WP \\ \hline
\comp a \comp c \comp d & x &   & x &   &   &   &   &   & x(4) \\
      a \comp b       d &   &   &   &   &   & x & x &   &    \\
      b       c       d &   &   &   & x &   &   &   & x & x(7) \\
      a       c       d &   &   &   &   &   &   & x & x &    \\
\comp b \comp c         & x &   &   &   & x &   &   &   &    \\
\comp b \comp d         & x & x &   &   & x & x &   &   & x(2) \\
\end{tabular}
\end{center}

\textbf{4. minimale Restüberdeckung}

Nach Herausstreichen wesentlicher Primimplikanten und Kürzen abgedeckter Minterme bleibt:
\begin{center}
\begin{tabular}{>{$}l<{$}|cccccccc}
\text{Primimpl.}        &11   \\ \hline
      a \comp b       d & x   \\
      a       c       d & x   \\
\end{tabular}
\end{center}
Das kann direkt als Alternative angegeben werden: 
$\left\{\genfrac{}{}{}{}{a\comp b d}{a c d}\right\}$

\bigskip

\textbf{5. Lösung}

%Wesentliche Primimplikanten:
%\begin{itemize}
	%\item $\comp a \comp c \comp d$
  %\item $bcd$
  %\item $\comp b \comp d$
%\end{itemize}
%
%Restüberdeckung: $\left\{\genfrac{}{}{}{}{a\comp b d}{a c d}\right\} \rightarrow acd$ 
%
%\smallskip

$f(a,b,c,d) = \comp a \comp c \comp d + bcd + \comp b \comp d + acd$

\subex{KV Visualisierung}
ToDo

\ex{Quine-McCluskey mit Petrick}{21}
\begin{align*}
f(a,b,c,d) = \sum\limits^{15}_{i=0} \text{\bf Min}^{4}_i \cdot r_i, &\text{ mit } r_i=1 \text{ für } i = \left\{0,1,2,4,6,7,8,9,10,11,15\right\}
\end{align*}
%
%\begin{center}
%\begin{tabular}{CCCC|R}
%a & b & c & d & f \\ \hline
%0 & 0 & 0 & 0 & 1 \\
%0 & 0 & 0 & 1 & 1 \\
%0 & 0 & 1 & 0 & 1 \\
%0 & 0 & 1 & 1 & 0 \\
%0 & 1 & 0 & 0 & 1 \\
%0 & 1 & 0 & 1 & 0 \\
%0 & 1 & 1 & 0 & 1 \\
%0 & 1 & 1 & 1 & 1 \\
%1 & 0 & 0 & 0 & 1 \\
%1 & 0 & 0 & 1 & 1 \\
%1 & 0 & 1 & 0 & 1 \\
%1 & 0 & 1 & 1 & 1 \\
%1 & 1 & 0 & 0 & 0 \\
%1 & 1 & 0 & 1 & 0 \\
%1 & 1 & 1 & 0 & 0 \\
%1 & 1 & 1 & 1 & 1 \\
%\end{tabular}
%\end{center}

\textbf{1. disjunktive kanonische Normalform}

$f = \comp a \comp b \comp c \comp d
   + \comp a \comp b \comp c       d
   + \comp a \comp b       c \comp d
   + \comp a       b \comp c \comp d
   + \comp a       b       c \comp d
   + \comp a       b       c       d
   +       a \comp b \comp c \comp d
   +       a \comp b \comp c       d
   +       a \comp b       c \comp d
   +       a \comp b       c       d
   +       a       b       c       d$
  
\bigskip

\textbf{2. Primimplikanten}

\begin{center}
\begin{tabular}{ c | c >{$}c<{$} | c >{$}c<{$} | c >{$}c<{$}}
Klasse & \# & \text{Minterme} & \# & \text{Verschm. MinT} & \# & \text{Verschm. MinT}\\\hline
\multirow{1}{*}{$K_0$} & 0  & \comp a \comp b \comp c \comp d & 0,1   & \comp a \comp b \comp c         & 0,1,8,9 & \comp b \comp c \\
                       &    &                                 & 0,2   & \comp a \comp b         \comp d & 0,2,4,6 & \comp a \comp d \\
                       &    &                                 &       &                                 & 0,2,8,10& \comp b \comp d \\
                       &    &                                 & 0,4   & \comp a         \comp c \comp d & 0,4,2,6 & \comp a \comp d \\
                       &    &                                 & 0,8   &         \comp b \comp c \comp d & 0,8,1,9 & \comp b \comp c\\
                       &    &                                 & 0,8   &         \comp b \comp c \comp d & 0,8,2,10& \comp b \comp d\\

\multirow{1}{*}{$K_1$} & 1  & \comp a \comp b \comp c       d & 1,9   &         \comp b \comp c       d & &\\
                       & 2  & \comp a \comp b       c \comp d & 2,6   & \comp a               c \comp d & &\\
                       &    &                                 & 2,10  &         \comp b       c \comp d & &\\
                       & 4  & \comp a       b \comp c \comp d & 4,6   & \comp a       b         \comp d & &\\
                       & 8  &       a \comp b \comp c \comp d & 8,9   &       a \comp b \comp c         & 8,9,10,11 & a \comp b\\
                       &    &                                 & 8,10  &       a \comp b         \comp d & 8,10,9,11 & a \comp b \\
                      
\multirow{1}{*}{$K_2$} & 6  & \comp a       b       c \comp d & 6,7   & \comp a       b       c         & &\\
                       & 9  &       a \comp b \comp c       d & 9,11  &       a \comp b               d & &\\
                       & 10 &       a \comp b       c \comp d & 10,11 &       a \comp b       c         & &\\
                      
\multirow{1}{*}{$K_3$} & 7  & \comp a       b       c       d & 7,15  &               b       c       d & &\\
                       & 11 &       a \comp b       c       d & 11,15 &       a               c       d & &\\
                      
\multirow{1}{*}{$K_4$} & 15 &       a       b       c       d &       &                                 & &
\end{tabular}
\end{center}

\textbf{3. wesentliche Primimplikanten}
\begin{center}
\begin{tabular}{>{$}l<{$}|ccccccccccc|c}
\text{Primimpl.}        & 0 & 1 & 2 & 4 & 6 & 7 & 8 & 9 &10 &11 &15 & WP   \\ \hline
\comp b \comp c         & x & x &   &   &   &   & x & x &   &   &   & x(1) \\
\comp a \comp d         & x &   & x & x & x &   &   &   &   &   &   & x(4) \\
\comp b \comp d         & x &   & x &   &   &   & x &   & x &   &   &      \\
      a \comp b         &   &   &   &   &   &   & x & x & x & x &   &      \\
\comp a b c             &   &   &   &   & x & x &   &   &   &   &   &      \\
bcd                     &   &   &   &   &   & x &   &   &   &   & x &      \\
acd                     &   &   &   &   &   &   &   &   &   & x & x &      
\end{tabular}
\end{center}

\textbf{4. Restüberdeckung}
\begin{center}
\begin{tabular}{>{$}l<{$}>{$}l<{$}|cccc|}
    &                         & 7 &10 &11 &15 \\ \hline
P_1 & \comp b \comp d         &   & x &   &   \\
P_2 &       a \comp b         &   & x & x &   \\
P_3 & \comp a b c             & x &   &   &   \\
P_4 & bcd                     & x &   &   & x \\
P_5 & acd                     &   &   & x & x 
\end{tabular}
\end{center}

\begin{align*}
 &(P_3+P_4)(P_1+P_2)(P_2+P_5)(P_4+P_5) \\
=&(P_3P_1+P_3P_2+P_4P_1+P_4P_2)(P_2+P_5)(P_4+P_5) \\
=&(P_3P_1P_2+P_3P_2P_2+P_4P_1P_2+P_4P_2P_2+P_3P_1P_5+P_3P_2P_5+P_4P_1P_5+P_4P_2P_5)(P_4+P_5) \\
=&(P_3P_1P_2+P_3P_2+P_4P_1P_2+P_4P_2+P_3P_1P_5+P_3P_2P_5+P_4P_1P_5+P_4P_2P_5)(P_4+P_5) \\
=&P_3P_1P_2P_4+P_3P_2P_4+P_4P_1P_2P_4+P_4P_2P_4+P_3P_1P_5P_4+P_3P_2P_5P_4+P_4P_1P_5P_4\\
+&P_4P_2P_5P_4+P_3P_1P_2P_5+P_3P_2P_5+P_4P_1P_2P_5+P_4P_2P_5+P_3P_1P_5P_5+P_3P_2P_5P_5\\
+&P_4P_1P_5P_5+P_4P_2P_5P_5 \\
=&P_1P_2P_3P_4+P_2P_3P_4+P_1P_2P_4+P_2P_4+P_1P_3P_4P_5+P_2P_3P_4P_5+P_1P_4P_5+P_2P_4P_5 \\
+&P_1P_3P_5+P_2P_3P_5+P_1P_2P_4P_5+P_2P_4P_5+P_1P_3P_5+P_2P_3P_5+P_1P_4P_5+P_2P_4P_5 \\
\Rightarrow&P_2P_4
\end{align*}
$P_2P_4\rightarrow a\comp b + bcd$

\bigskip

\textbf{5. Lösung}

$f(a,b,c,d)=\comp b \comp c + \comp a \comp d + a \comp b + bcd$.


\ex{Koppelterme}{6 + 2 + 6 + 8 + 4 + 2 = 28}


\ex{Vereinfachung der 10-er Minutenstelle}{21}

\begin{center}

\begin{tabular}{ccc}
$a = x_1 + x_0 x_2 + \comp x_2 \comp x_0$ & $b = \comp x_0 + \comp x_2$ & $c = \comp x_1 + x_0$ \\\bigskip

\begin{tikzpicture}
  \matrix (m)[kv map]{
     1 & 1 & 1 & 0 \\
     0 & 1 & 1 & 1 \\
  };\kvmeta
  \kvgrp{1-1}{1-2}
  \kvgrp[dotted]{1-2}{2-3}
  \kvgrp[dashed]{2-3}{2-4}
\end{tikzpicture} & 

\begin{tikzpicture}
  \matrix (m)[kv map]{
     1 & 1 & 1 & 1 \\
     1 & 1 & 0 & 0 \\
  };\kvmeta
  \kvgrp{1-1}{1-4}
  \kvgrp[dotted]{1-1}{2-2}
\end{tikzpicture} & 

\begin{tikzpicture}
  \matrix (m)[kv map]{
     1 & 0 & 1 & 1 \\
     1 & 0 & 1 & 1 \\
  };\kvmeta
  \kvgrp[dashed]{1-3}{2-4}
  \kvgrpclp{1-1}{2-1}{1-4}{2-4}
\end{tikzpicture} \\

$d = x_1 + x_0 x_2 + \comp x_2 \comp x_0$ & $e = \comp x_2 \comp x_0$ & $f = x_2 + \comp x_0 \comp x_1$ \\\bigskip

\begin{tikzpicture}
  \matrix (m)[kv map]{
     1 & 1 & 1 & 0 \\
     0 & 1 & 1 & 1 \\
  };\kvmeta
  \kvgrp{1-1}{1-2}
  \kvgrp[dotted]{1-2}{2-3}
  \kvgrp[dashed]{2-3}{2-4}
\end{tikzpicture} & 

\begin{tikzpicture}
  \matrix (m)[kv map]{
     1 & 1 & 0 & 0 \\
     0 & 0 & 0 & 0 \\
  };\kvmeta
  \kvgrp{1-1}{1-2}
\end{tikzpicture} & 

\begin{tikzpicture}
  \matrix (m)[kv map]{
     1 & 0 & 0 & 0 \\
     1 & 1 & 1 & 1 \\
  };\kvmeta
  \kvgrp{1-1}{2-1}
  \kvgrp{2-1}{2-4}
\end{tikzpicture} \\

$g = x_1 + x_2$ & & \\

\begin{tikzpicture}
  \matrix (m)[kv map]{
     0 & 1 & 1 & 0 \\
     1 & 1 & 1 & 1 \\
  };\kvmeta
  \kvgrp{1-2}{2-3}
  \kvgrp[dotted]{2-1}{2-4}
\end{tikzpicture} & & \\

\end{tabular}
\end{center}



\end{document}
