\documentclass{CInf_practice}

\sheet{4}{Quine-McCluskey und Koppelterme}


\usepackage{multicol}
\usetikzlibrary{matrix}
\tikzstyle{highlight}=[densely dotted,line width=.7pt,draw,rounded corners=3pt]
\tikzstyle{matrix node}=[anchor=center,draw,shape=rectangle,text width=1.5em,minimum height=1.5em,text centered,font=\scriptsize]
\tikzstyle{kv map}=[nodes={matrix node}, matrix of nodes, inner sep=0pt, outer sep=0pt, row sep=-.5pt, column sep=-.5pt]

\newcommand{\kvmeta}{
\draw[|-|] ($(m-2-1.north west) + (-.2em,0)$) -- node[left] () {$\scriptstyle x_0$} ++(0,-1.5em);
\draw[|-|] ($(m-1-2.north west) + (0,.2em)$) -- node[above] () {$\scriptstyle x_1$} ++(3em,0);
\draw[|-|] ($(m-1-3.north west) + (0,1.4em)$) -- node[above] () {$\scriptstyle x_2$} ++(3em,0);
\clip (m.north west) rectangle (m.south east);
}

\newcommand{\kvgroup}[5][]{\kvgrp[#1]{#2-#3}{#4-#5}}
\newcommand{\kvgrp}[3][]{\draw[highlight,#1] ($(m-#2.north west) + (2pt,-2pt)$) rectangle ($(m-#3.south east) + (-2pt,2pt)$);}
\newcommand{\kvgrpclp}[5][]{
\draw[highlight,#1] ($(m-#2.north west) + (-2pt,-2pt)$) rectangle ($(m-#3.south east) + (-2pt,2pt)$);
\draw[highlight,#1] ($(m-#4.north west) + (2pt,-2pt)$) rectangle ($(m-#5.south east) + (2pt,2pt)$);
}
%\def\kvgroup#1#2#3#4#5{\kvgrp{#1}{#2-#3}{#4-#5}}

\begin{document}
	

\cinftitle

\ex{Minimierung nach Quine-McCluskey}{20 + 10 = 30}
\subex{Minimierung}

\begin{align*}
f(a,b,c,d) = \sum\limits^{15}_{i=0} \text{\bf Min}^{4}_i \cdot r_i, &\text{ mit } r_i=1 \text{ für } i = \left\{0,2,4,7,8,10,11,15\right\}\\
& \text{ und mit Don't Care für } i=\left\{1,9\right\}
\end{align*}

%\begin{center}
%\begin{tabular}{CCCC|R}
%a & b & c & d & f \\ \hline
%0 & 0 & 0 & 0 & 1 \\
%0 & 0 & 0 & 1 & X \\
%0 & 0 & 1 & 0 & 1 \\
%0 & 0 & 1 & 1 & 0 \\
%0 & 1 & 0 & 0 & 1 \\
%0 & 1 & 0 & 1 & 0 \\
%0 & 1 & 1 & 0 & 0 \\
%0 & 1 & 1 & 1 & 1 \\
%1 & 0 & 0 & 0 & 1 \\
%1 & 0 & 0 & 1 & X \\
%1 & 0 & 1 & 0 & 1 \\
%1 & 0 & 1 & 1 & 1 \\
%1 & 1 & 0 & 0 & 0 \\
%1 & 1 & 0 & 1 & 0 \\
%1 & 1 & 1 & 0 & 0 \\
%1 & 1 & 1 & 1 & 1 \\
%\end{tabular}
%\end{center}

\textbf{1. disjunktive kanonische Normalform}

$f = \comp a \comp b \comp c \comp d
   + \comp a \comp b \comp c       d
   + \comp a \comp b       c \comp d 
   + \comp a       b \comp c \comp d
   + \comp a       b       c       d 
   +       a \comp b \comp c \comp d
   +       a \comp b \comp c       d   
   +       a \comp b       c \comp d 
   +       a \comp b       c       d 
   +       a       b       c       d$

\bigskip

\textbf{2. Primimplikanten}

\begin{center}
\begin{tabular}{ c | c >{$}c<{$} | c >{$}c<{$} | c >{$}c<{$}}
Klasse & \# & \text{Minterme} & \# & \text{Verschm. MinT} & \# & \text{Verschm. MinT}\\\hline
\multirow{1}{*}{$K_0$} & 0  & \comp a \comp b \comp c \comp d & 0,1   & \comp a \comp b \comp c         & 0,1-8,9  & \comp b \comp c \\
                       &    &                                 & 0,2   & \comp a \comp b         \comp d &&\\
                       &    &                                 & 0,4   & \comp a         \comp c \comp d &&\\
                       &    &                                 & 0,8   &         \comp b \comp c \comp d & 0,8-1,9  & \comp b \comp c\\
                       &    &                                 &       &                                 & 0,8-2,10 & \comp b \comp d\\
                      
\multirow{1}{*}{$K_1$} & 1  & \comp a \comp b \comp c       d & 1,9   &         \comp b \comp c       d &&\\
                       & 2  & \comp a \comp b       c \comp d & 2,10  &         \comp b       c \comp d &&\\
                       & 4  & \comp a       b \comp c \comp d &       & &&\\
                       & 8  &       a \comp b \comp c \comp d & 8,9   &       a \comp b \comp c         &&\\
                       &    &                                 & 8,10  &       a \comp b               d &&\\
                      
\multirow{1}{*}{$K_2$} & 9  &       a \comp b \comp c       d & 9,11  &       a \comp b               d &&\\
                       & 10 &       a \comp b       c \comp d & 10,11 &       a \comp b               d &&\\
                      
\multirow{1}{*}{$K_3$} & 7  & \comp a       b       c       d & 7,15  &               b       c       d &&\\
                       & 11 &       a \comp b       c       d & 11,15 &       a               c       d &&\\
                      
\multirow{1}{*}{$K_4$} & 15 &       a       b       c       d &       & &&\\
\end{tabular}
\end{center}

\textbf{3. wesentliche Primimplikanten}

\begin{center}
\begin{tabular}{>{$}l<{$}|cccccccc|c}
\text{Primimpl.}        & 0 & 2 & 4 & 7 & 8 &10 &11 &15 & WP \\ \hline
\comp a \comp c \comp d & x &   & x &   &   &   &   &   & x(4) \\
      a \comp b       d &   &   &   &   &   & x & x &   &    \\
      b       c       d &   &   &   & x &   &   &   & x & x(7) \\
      a       c       d &   &   &   &   &   &   & x & x &    \\
\comp b \comp c         & x &   &   &   & x &   &   &   &    \\
\comp b \comp d         & x & x &   &   & x & x &   &   & x(2) \\
\end{tabular}
\end{center}

\textbf{4. minimale Restüberdeckung}

Nach Herausstreichen wesentlicher Primimplikanten und Kürzen abgedeckter Minterme bleibt:
\begin{center}
\begin{tabular}{>{$}l<{$}|cccccccc}
\text{Primimpl.}        &11   \\ \hline
      a \comp b       d & x   \\
      a       c       d & x   \\
\end{tabular}
\end{center}
Das kann direkt als Alternative angegeben werden: 
$\left\{\genfrac{}{}{}{}{a\comp b d}{a c d}\right\}$

\bigskip

\textbf{5. Lösung}

%Wesentliche Primimplikanten:
%\begin{itemize}
	%\item $\comp a \comp c \comp d$
  %\item $bcd$
  %\item $\comp b \comp d$
%\end{itemize}
%
%Restüberdeckung: $\left\{\genfrac{}{}{}{}{a\comp b d}{a c d}\right\} \rightarrow acd$ 
%
%\smallskip

$f(a,b,c,d) = \comp a \comp c \comp d + bcd + \comp b \comp d + acd$

\subex{KV Visualisierung}
ToDo

\ex{Quine-McCluskey mit Petrick}{21}
\begin{align*}
f(a,b,c,d) = \sum\limits^{15}_{i=0} \text{\bf Min}^{4}_i \cdot r_i, &\text{ mit } r_i=1 \text{ für } i = \left\{0,1,2,4,6,7,8,9,10,11,15\right\}
\end{align*}
%
%\begin{center}
%\begin{tabular}{CCCC|R}
%a & b & c & d & f \\ \hline
%0 & 0 & 0 & 0 & 1 \\
%0 & 0 & 0 & 1 & 1 \\
%0 & 0 & 1 & 0 & 1 \\
%0 & 0 & 1 & 1 & 0 \\
%0 & 1 & 0 & 0 & 1 \\
%0 & 1 & 0 & 1 & 0 \\
%0 & 1 & 1 & 0 & 1 \\
%0 & 1 & 1 & 1 & 1 \\
%1 & 0 & 0 & 0 & 1 \\
%1 & 0 & 0 & 1 & 1 \\
%1 & 0 & 1 & 0 & 1 \\
%1 & 0 & 1 & 1 & 1 \\
%1 & 1 & 0 & 0 & 0 \\
%1 & 1 & 0 & 1 & 0 \\
%1 & 1 & 1 & 0 & 0 \\
%1 & 1 & 1 & 1 & 1 \\
%\end{tabular}
%\end{center}

\textbf{1. disjunktive kanonische Normalform}

$f = \comp a \comp b \comp c \comp d
   + \comp a \comp b \comp c       d
   + \comp a \comp b       c \comp d
   + \comp a       b \comp c \comp d
   + \comp a       b       c \comp d
   + \comp a       b       c       d
   +       a \comp b \comp c \comp d
   +       a \comp b \comp c       d
   +       a \comp b       c \comp d
   +       a \comp b       c       d
   +       a       b       c       d$
  
\bigskip

\textbf{2. Primimplikanten}

\begin{center}
\begin{tabular}{ c | c >{$}c<{$} | c >{$}c<{$} | c >{$}c<{$}}
Klasse & \# & \text{Minterme} & \# & \text{Verschm. MinT} & \# & \text{Verschm. MinT}\\\hline
\multirow{1}{*}{$K_0$} & 0  & \comp a \comp b \comp c \comp d & 0,1   & \comp a \comp b \comp c         & 0,1,8,9 & \comp b \comp c \\
                       &    &                                 & 0,2   & \comp a \comp b         \comp d & 0,2,4,6 & \comp a \comp d \\
                       &    &                                 &       &                                 & 0,2,8,10& \comp b \comp d \\
                       &    &                                 & 0,4   & \comp a         \comp c \comp d & 0,4,2,6 & \comp a \comp d \\
                       &    &                                 & 0,8   &         \comp b \comp c \comp d & 0,8,1,9 & \comp b \comp c\\
                       &    &                                 & 0,8   &         \comp b \comp c \comp d & 0,8,2,10& \comp b \comp d\\

\multirow{1}{*}{$K_1$} & 1  & \comp a \comp b \comp c       d & 1,9   &         \comp b \comp c       d & &\\
                       & 2  & \comp a \comp b       c \comp d & 2,6   & \comp a               c \comp d & &\\
                       &    &                                 & 2,10  &         \comp b       c \comp d & &\\
                       & 4  & \comp a       b \comp c \comp d & 4,6   & \comp a       b         \comp d & &\\
                       & 8  &       a \comp b \comp c \comp d & 8,9   &       a \comp b \comp c         & 8,9,10,11 & a \comp b\\
                       &    &                                 & 8,10  &       a \comp b         \comp d & 8,10,9,11 & a \comp b \\
                      
\multirow{1}{*}{$K_2$} & 6  & \comp a       b       c \comp d & 6,7   & \comp a       b       c         & &\\
                       & 9  &       a \comp b \comp c       d & 9,11  &       a \comp b               d & &\\
                       & 10 &       a \comp b       c \comp d & 10,11 &       a \comp b       c         & &\\
                      
\multirow{1}{*}{$K_3$} & 7  & \comp a       b       c       d & 7,15  &               b       c       d & &\\
                       & 11 &       a \comp b       c       d & 11,15 &       a               c       d & &\\
                      
\multirow{1}{*}{$K_4$} & 15 &       a       b       c       d &       &                                 & &
\end{tabular}
\end{center}

\textbf{3. wesentliche Primimplikanten}
\begin{center}
\begin{tabular}{>{$}l<{$}|ccccccccccc|c}
\text{Primimpl.}        & 0 & 1 & 2 & 4 & 6 & 7 & 8 & 9 &10 &11 &15 & WP   \\ \hline
\comp b \comp c         & x & x &   &   &   &   & x & x &   &   &   & x(1) \\
\comp a \comp d         & x &   & x & x & x &   &   &   &   &   &   & x(4) \\
\comp b \comp d         & x &   & x &   &   &   & x &   & x &   &   &      \\
      a \comp b         &   &   &   &   &   &   & x & x & x & x &   &      \\
\comp a b c             &   &   &   &   & x & x &   &   &   &   &   &      \\
bcd                     &   &   &   &   &   & x &   &   &   &   & x &      \\
acd                     &   &   &   &   &   &   &   &   &   & x & x &      
\end{tabular}
\end{center}

\textbf{4. Restüberdeckung}
\begin{center}
\begin{tabular}{>{$}l<{$}>{$}l<{$}|cccc|}
    &                         & 7 &10 &11 &15 \\ \hline
P_1 & \comp b \comp d         &   & x &   &   \\
P_2 &       a \comp b         &   & x & x &   \\
P_3 & \comp a b c             & x &   &   &   \\
P_4 & bcd                     & x &   &   & x \\
P_5 & acd                     &   &   & x & x 
\end{tabular}
\end{center}

\begin{align*}
 &(P_3+P_4)(P_1+P_2)(P_2+P_5)(P_4+P_5) \\
=&(P_3P_1+P_3P_2+P_4P_1+P_4P_2)(P_2+P_5)(P_4+P_5) \\
=&(P_3P_1P_2+P_3P_2P_2+P_4P_1P_2+P_4P_2P_2+P_3P_1P_5+P_3P_2P_5+P_4P_1P_5+P_4P_2P_5)(P_4+P_5) \\
=&(P_3P_1P_2+P_3P_2+P_4P_1P_2+P_4P_2+P_3P_1P_5+P_3P_2P_5+P_4P_1P_5+P_4P_2P_5)(P_4+P_5) \\
=&P_3P_1P_2P_4+P_3P_2P_4+P_4P_1P_2P_4+P_4P_2P_4+P_3P_1P_5P_4+P_3P_2P_5P_4+P_4P_1P_5P_4\\
+&P_4P_2P_5P_4+P_3P_1P_2P_5+P_3P_2P_5+P_4P_1P_2P_5+P_4P_2P_5+P_3P_1P_5P_5+P_3P_2P_5P_5\\
+&P_4P_1P_5P_5+P_4P_2P_5P_5 \\
=&P_1P_2P_3P_4+P_2P_3P_4+P_1P_2P_4+P_2P_4+P_1P_3P_4P_5+P_2P_3P_4P_5+P_1P_4P_5+P_2P_4P_5 \\
+&P_1P_3P_5+P_2P_3P_5+P_1P_2P_4P_5+P_2P_4P_5+P_1P_3P_5+P_2P_3P_5+P_1P_4P_5+P_2P_4P_5 \\
\Rightarrow&P_2P_4
\end{align*}
$P_2P_4\rightarrow a\comp b + bcd$

\bigskip

\textbf{5. Lösung}

$f(a,b,c,d)=\comp b \comp c + \comp a \comp d + a \comp b + bcd$.


\ex{Koppelterme}{6 + 2 + 6 + 8 + 4 + 2 = 28}


\ex{Vereinfachung der 10-er Minutenstelle}{21}

\begin{center}

\begin{tabular}{ccc}
$a = x_1 + x_0 x_2 + \comp x_2 \comp x_0$ & $b = \comp x_0 + \comp x_2$ & $c = \comp x_1 + x_0$ \\\bigskip

\begin{tikzpicture}
  \matrix (m)[kv map]{
     1 & 1 & 1 & 0 \\
     0 & 1 & 1 & 1 \\
  };\kvmeta
  \kvgrp{1-1}{1-2}
  \kvgrp[dotted]{1-2}{2-3}
  \kvgrp[dashed]{2-3}{2-4}
\end{tikzpicture} & 

\begin{tikzpicture}
  \matrix (m)[kv map]{
     1 & 1 & 1 & 1 \\
     1 & 1 & 0 & 0 \\
  };\kvmeta
  \kvgrp{1-1}{1-4}
  \kvgrp[dotted]{1-1}{2-2}
\end{tikzpicture} & 

\begin{tikzpicture}
  \matrix (m)[kv map]{
     1 & 0 & 1 & 1 \\
     1 & 0 & 1 & 1 \\
  };\kvmeta
  \kvgrp[dashed]{1-3}{2-4}
  \kvgrpclp{1-1}{2-1}{1-4}{2-4}
\end{tikzpicture} \\

$d = x_1 + x_0 x_2 + \comp x_2 \comp x_0$ & $e = \comp x_2 \comp x_0$ & $f = x_2 + \comp x_0 \comp x_1$ \\\bigskip

\begin{tikzpicture}
  \matrix (m)[kv map]{
     1 & 1 & 1 & 0 \\
     0 & 1 & 1 & 1 \\
  };\kvmeta
  \kvgrp{1-1}{1-2}
  \kvgrp[dotted]{1-2}{2-3}
  \kvgrp[dashed]{2-3}{2-4}
\end{tikzpicture} & 

\begin{tikzpicture}
  \matrix (m)[kv map]{
     1 & 1 & 0 & 0 \\
     0 & 0 & 0 & 0 \\
  };\kvmeta
  \kvgrp{1-1}{1-2}
\end{tikzpicture} & 

\begin{tikzpicture}
  \matrix (m)[kv map]{
     1 & 0 & 0 & 0 \\
     1 & 1 & 1 & 1 \\
  };\kvmeta
  \kvgrp{1-1}{2-1}
  \kvgrp{2-1}{2-4}
\end{tikzpicture} \\

$g = x_1 + x_2$ & & \\

\begin{tikzpicture}
  \matrix (m)[kv map]{
     0 & 1 & 1 & 0 \\
     1 & 1 & 1 & 1 \\
  };\kvmeta
  \kvgrp{1-2}{2-3}
  \kvgrp[dotted]{2-1}{2-4}
\end{tikzpicture} & & \\

\end{tabular}
\end{center}

\subex{Koppelterme}

\begin{tabular}{>{$}l<{$} >{$}l<{$} r}
\text{\bf Funktionen} & \text{\bf Koppelterme} \\
a\times d & x_1+x_0x_2+\comp x_0 \comp x_2 \\
a\times e & \comp x_0 \comp x_2 \\
a\times g & x_1 \\
d\times e & \comp x_0 \comp x_2 \\
d\times g & x_1 \\
f\times g & x_2 \\
\end{tabular}

\begin{tabular}{>{$}l<{$} >{$}l<{$}}
\text{\bf Funktionen} & \text{\bf Koppelterme} \\
(a\times d) \times (a\times e) & \comp x_0 \comp x_2 \\
(a\times d) \times (a\times g) & x_1 \\
(a\times d) \times (d\times e) & \comp x_0 \comp x_2 \\
(a\times d) \times (d\times g) & x_1 \\
(a\times e) \times (d\times e) & \comp x_0 \comp x_2 \\
(a\times g) \times (d\times g) & x_1 \\
\end{tabular}

\subex{Schaltung}
\begin{center}
\begin{tikzpicture}
  \def\xOffset{.45}
  \def\circSize{1.7pt}
  \newcommand{\module}[5][]{ % \module[<name>]{<text>}{<(x,y)>}{<width>}{<height>}
    \draw #3 rectangle node (#1) {#2}  ++(#4,-#5);
    \node[anchor=south west] at($#3 + (0,-.25*#5)$) (#1_top) {};
    \node[anchor=west]       at($#3 + (0,-.5*#5)$)  (#1_middle) {};
    \node[anchor=north west] at($#3 + (0,-.75*#5)$) (#1_bottom) {};
    \node[anchor=east]       at($(#1_middle) + (.85*#4, 0)$)(#1_out) {};
  }
  \def\solder#1#2#3{    % \solder{<line>}{<module>}{<position>}
    \draw[-] (#2_#3 -| #1) -- (#2_#3);
    \fill[black] (#2_#3 -| #1) circle (\circSize);
  }
  \def\solderNeg#1#2#3{ % \solderNeg{<line>}{<module>}{<position>}
    \draw[-o] (#2_#3 -| #1) -- (#2_#3);
    \fill[black] (#2_#3 -| #1) circle (\circSize);
  }
  \newcommand{\solderSplit}[2][]{
    \node[coordinate] at#2 (#1_) {};
    \node[coordinate] at#2 (#1) {};
    \fill[black] ($(#1)+(\xOffset,0)$) circle (\circSize);
    %\draw[-] (#1 -| #2) -- (#2);
    %\fill[black] (#1 -| #2) circle (\circSize);
  }
  \def\connectOut#1#2#3{
    \draw[-] (#1_out) -| ++(\xOffset,0) |- (#2_#3);
  }
  \def\connect#1#2{
    \draw[-] (#1) -| ++(\xOffset,0) |- (#2);
  }
  
  \def\width{.7}
  \def\height{.8}
  
  % leading lines
  \draw (0,   14) node[above] (x0) {$x_0$} -- (0,   0);
  \draw (0.5, 14) node[above] (x1) {$x_1$} -- (0.5, 0);
  \draw (1,   14) node[above] (x2) {$x_2$} -- (1,   0);
  
  % end nodes
  \foreach \l/\y in {f/6,g/5,a/4,d/3,e/2,b/1,c/0}
    \node[anchor=west] at(7, 0.5+2*\y) (\l) {\l};
  
  % f
  \module[and-f]{\&}{(1.5,13.5)}{\width}{\height}
  \module[or-f]{$\ge$}{(3.5,13.5)}{\width}{2*\height}
  \solderNeg{x0}{and-f}{top}
  \solderNeg{x1}{and-f}{bottom}
  \connectOut{and-f}{or-f}{top}
  \connect{or-f_out}{f}

  % g
  \module[or-g]{$\ge$}{(3.5,10.5)}{\width}{\height}
  \connect{or-g_out}{g}
  
  % f/g
  \solderSplit[split-f-g]{(2.5, 11.5)}
  \solder{x2}{split-f-g}{}
  \connect{split-f-g}{or-g_top}
  \connect{split-f-g}{or-f_bottom}
  
  % a
  \module[or-a]{$\ge$}{(4.5, 9.5)}{\width}{4*\height}
  \module[and-a1]{\&}{(2.5, 8.5)}{\width}{\height}
  \module[and-a2]{\&}{(2.5, 7.5)}{\width}{\height}
  
  \solder{x0}{and-a1}{top}
  \solder{x2}{and-a1}{bottom}
  \solderNeg{x0}{and-a2}{top}
  \solderNeg{x2}{and-a2}{bottom}
  \connectOut{and-a1}{or-a}{middle}
  
  % e/a
  %\connectOut{and-a2}{or-a}{bottom}
  \solderSplit[split-a-e]{(3.5,5.5)}
  \connectOut{and-a2}{split-a-e}{}
  \connect{split-a-e}{or-a_bottom}
  \connect{split-a-e}{e}
  
  % a/d
  \solderSplit[split-a-d]{(5.5,7.5)}
  \connect{split-a-d}{a}
  \connect{split-a-d}{d}
  \connect{or-a_out}{split-a-d}
  
  % g/a
  \solderSplit[split-g-a]{(2.5, 9.5)}
  \solder{x1}{split-g-a}{}
  \connect{split-g-a}{or-g_bottom}
  \connect{split-g-a}{or-a_top}
  
  % b
  \module[or-b]{$\ge$}{(2.5,3)}{\width}{\height}
  \solderNeg{x0}{or-b}{top}
  \solderNeg{x2}{or-b}{bottom}
  \connect{or-b}{b}
  
  % c
  \module[or-c]{$\ge$}{(2.5,1)}{\width}{\height}
  \solder{x0}{or-c}{top}
  \solder{x2}{or-c}{bottom}
  \connect{or-c}{c}
  
  
\end{tikzpicture}
\end{center}

\end{document}
