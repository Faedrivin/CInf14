\documentclass{CInf_practice}

\sheet{12}{Stoppuhr auf dem ATmega16}

\lstset{language={[avr8avra]Assembler},
        morecomment=[n]{/*}{*/},
        morecomment=[l]{//},
     }
\usepackage{marvosym}

\begin{document}
\cinftitle

Das gesamte Programm befindet sich am Ende der Abgabe.
Für die einzelnen Teilaufgaben ist jeweils der wichtigste Codeabschnitt
direkt erwähnt. Dabei sind aber ggf. Sprungadressen, Compilerdirektiven u.ä.
nicht direkt aufgeführt, wenn sie an anderer Stelle im Code stehen. Der gesamte
Code am Ende hält diese Befehle dann bereit.

Die Header-Datei des letzten Aufgabenblatts wurde hier wieder verwendet, sie 
befindet sich noch nach dem Programmcode ganz am Ende.

\ex{Systemstruktur}{6 + 4}

\subex{Modulbild}

\subex{Registerbelegungsplan}
\begin{tabular}{l|l|l}
\bf Register & \bf Name & \bf Funktion \\\hline
R16 & tmp & Temporäres Register, universell eingesetzt \\
R17 & time & aktuelle Zeit der Stoppuhr, hält kompakten BCD-Wert \\
X & X & Adressregister, für Table-Initialisierung und Lookups
\end{tabular}

\ex{Konfiguration der I/O-Ports}{10}
\lstinputlisting[linerange={58-85}]{stopwatch.asm}


\ex{Ausgabe der 7-Segment-Codes}{10}
\lstinputlisting[linerange={118-153,181-207}]{stopwatch.asm}


\ex{Hauptprogramm}{10}
\lstinputlisting[linerange={367-370,376-383,393-393}]{stopwatch.asm}


\ex{Polling vs. Interrupt}{4 + 10 + 4 + 6 + 4}

\subex{Polling}
Achtung, das Polling ist bereits auskommentiert, um den Änderungen aus 
Teilaufgabe d) gerecht zu werden.

\lstinputlisting[linerange={386-390}]{stopwatch.asm}

\subex{\texttt{COUNT\_UP}}
\lstinputlisting[linerange={222-248}]{stopwatch.asm}

\subex{\texttt{ISR\_CLOCK}}
\lstinputlisting[linerange={259-263,267-273}]{stopwatch.asm}

\subex{\texttt{INIT\_EXT\_INTERRUPT}}
Die Änderungen im Hauptprogramm sind das Auskommentieren des Pollings sowie
der Aufruf der \texttt{INIT\_EXT\_INTERRUPT}-Routine, wie im Code am Ende der 
Abgabe zu sehen ist.

Der Aufruf von \texttt{INIT\_EXT\_INTERRUPT} im Hauptprogramm ist ebenfalls 
auskommentiert, um Aufgabe 6b) gerecht zu werden.

\lstinputlisting[linerange={286-302}]{stopwatch.asm}

\subex{Vorteile Interrupts}
Die Stopp-Uhr ist genauer, da sie (mehr oder weniger) unabhängig vom 
restlichen Programm korrekt erhöht wird, und nicht nur dann, wenn gerade ein 
Polling ansteht.

\ex{T/C0-Interrupt}{6 + 14}
\subex{Prescale-Faktor}
Der Zusammenhang zwischen Prescale-Faktor $P$, Frequenz $F$ und Anzahl Bits 
des Zählregisters $B$ lässt sich wie folgt veranschaulichen:
\begin{align*}
\frac{F}{P} = 2^B Hz\\
\end{align*}
Daraus ergibt sich für $F = 2^{15} Hz$ und $B = 8$:
\begin{align*}
\frac{2^{15} Hz}{P} = 2^8 Hz \equiv P = \frac{2^{15} Hz}{2^8 Hz} = 2^7 = 128
\end{align*}
Die resultierende Auflösung ist dann $\frac{1 s}{2^7} = 0.0078125 s = 7.8125 ms$.

Der ATmega16 unterstützt die Prescale-Faktoren $8, 64, 256$ und $1024$. Wir 
müssen einen wählen, der größer ist als $128$, damit unser Zähler nicht 
ungewollt überläuft. $256$ bietet sich deshalb an. Der resultierende 
Vergleichswert ist $\frac{2^{15} Hz}{256 Hz} = 2^7 = 128$.

\subex{\texttt{INIT\_TCO\_INTERRUPT}}
Die weitere Änderung im Hauptprogramm ist, dass \texttt{TCNT0} zurückgesetzt
werden muss. Dies geschieht in der Interrupt-Serviceroutine \texttt{ISR\_CLOCK}.

Vorher war es außerdem nicht unbedingt notwendig, in \texttt{ISR\_CLOCK} den 
Programmstatus zu retten, nun wird aber das tmp-Register gebraucht und durch
den \texttt{CLR}-Befehl auch das \texttt{SREG} verändert. Es ist deshalb nötig,
nun den Programmstatus zu sichern und wieder herzustellen.

\lstinputlisting[linerange={321-342}]{stopwatch.asm}

\ex{Nebenläufigkeiten}{4 + 4 + 4}
Die Hauptprogramme werden unterschiedlich unterbrochen und weiterverarbeitet.
\texttt{cnt} steht stellvertretend für die Aufrufe der Interrupt-Handler und 
deren Unterprogramme.
Counter ist der T/C0-Zähler.

\begin{tabular}{ccc}
\subex{Polling}&
\subex{Externer INT0 Interrupt}&
\subex{Interner T/C0-Interrupt}\\
\tikz{
  \draw[->] (.5,7.5) -- (.5,7); 
  \draw[thin] (.25,7) rectangle node {\rotatebox{90}{Programm}} (.75,.5); 
  \draw[->] (.5,.5) -- (.5,0); 
} & 
\tikz{
  \draw[->] (2.5,7.5) -- (2.5,7); 
  \draw[thin] (2.25,7) rectangle node {\rotatebox{90}{Programm}} (2.75,.5); 
  \draw[->] (2.5,.5) -- (2.5,0); 
  
  \draw[->] (3.5,6.5) -- node[above] {\Lightning} (2.75,6);
  \draw[-] (2.75,6) -- ++(-.5,0);
  \draw[thin] (.75,5.5) rectangle node {\rotatebox{90}{cnt}} (1.25,2.5);
  \draw[->]  (2.25,6) -- (1.25,5.5);
  \draw[->] (1.25,2.5) -- (2.25,6);
} & 
\tikz{
  \draw[->] (2.5,7.5) -- (2.5,7); 
  \draw[thin] (2.25,7) rectangle node {\rotatebox{90}{Programm}} (2.75,.5); 
  \draw[->] (2.5,.5) -- (2.5,0);
  
  \draw[->] (4.5,7.5) -- (4.5,7); 
  \draw[thin] (4.25,7) rectangle node {\rotatebox{90}{Counter}} (4.75,.5); 
  \draw[->] (4.5,.5) -- (4.5,0);
  
  \draw[->] (4.25,3.5) -- node[above] {\Lightning} (2.75,4.5);
  \draw[-] (2.75,4.5) -- ++(-.5,0);
  \draw[thin] (.75,5.5) rectangle node {\rotatebox{90}{cnt}} (1.25,2.5);
  \draw[->]  (2.25,4.5) -- (1.25,5.5);
  \draw[->] (1.25,2.5) -- (2.25,4.5);
  
} \\
\end{tabular}

\newpage
{\large\textbf{Vollständiger Programmcode}}
\lstinputlisting{stopwatch.asm}

\bigskip

{\large\textbf{\texttt{init\_header.inc}}}
\lstinputlisting{init_header.inc}

\end{document}
