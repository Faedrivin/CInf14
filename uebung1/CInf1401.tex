\documentclass{CInf_practice}

\sheet{1}{Beispiel-CPU H6809}

\begin{document}

\cinftitle



\ex{Arithmetik auf Variablen}{12}



\ex{For-Schleife}{8+16=24}



\ex{Taschenrechner}{8+16+8=32}

\subex{Programmablauf}

\subex{Asemblerprogramm}

\begin{tabularx}{\textwidth}{|>{\ttfamily}l|>{\ttfamily}c|>{\ttfamily}c|>{\ttfamily}c|>{\ttfamily}l|>{\ttfamily}r|X|} % TODO: Solve with newcolumntype
   \hline
   \rmfamily\multirow{2}{*}{Adr (hex)} & \multicolumn{2}{c|}{\rmfamily M(Adr)
(hex)} & \rmfamily\multirow{2}{*}{Label} & \rmfamily\multirow{2}{*}{Opcode} &
   \rmfamily\multirow{2}{*}{Operand} &
   \multicolumn{1}{c|}{\rmfamily\multirow{2}{*}{Kommentare}} \\
                              & \footnotesize\rmfamily Opc. & \rmfamily\footnotesize Oper. & & & &\\\hline\hline
   woot & waat & weet & hoo & ha & he & thi s is a comment
   
\end{tabularx}


\ex{Codeumwandlung}{8+16+8=32}



\addex{Multiplikation von Variablen}



\end{document}
