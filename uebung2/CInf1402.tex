\documentclass{CInf_practice}

\sheet{2}{Beispiel-CPU H6809 und Schaltfunktionen}

\begin{document}

\cinftitle

\ex{Division mit Quotient und Rest}{6 + 12 + 6 = 24}

\ex{Addition von BCD-Zahlen}{6 + 12 = 18}

\ex{Absorptionsgesetze}{7 + 7 = 14}

\ex{Schaltfunktionen}{6 + 14 + 6 = 26}

\subex{Wertetabelle $f$ und $g$}

\begin{center}
  \begin{tabular}{cccc|c|c}
    \bf a & \bf b & \bf c & \bf d & \bf f & \bf g \\ \hline
    0 & 0 & 0 & 0 & 0 & 0 \\
    0 & 0 & 0 & 1 & 0 & 0 \\
    0 & 0 & 1 & 0 & 1 & 1 \\
    0 & 0 & 1 & 1 & 1 & 1 \\
    0 & 1 & 0 & 0 & 0 & 0 \\
    0 & 1 & 0 & 1 & 0 & 0 \\
    0 & 1 & 1 & 0 & 1 & 1 \\
    0 & 1 & 1 & 1 & 1 & 1 \\
    1 & 0 & 0 & 0 & 1 & 1 \\
    1 & 0 & 0 & 1 & 1 & 1 \\
    1 & 0 & 1 & 0 & 1 & 1 \\
    1 & 0 & 1 & 1 & 1 & 1 \\
    1 & 1 & 0 & 0 & 0 & 0 \\
    1 & 1 & 0 & 1 & 0 & 0 \\
    1 & 1 & 1 & 0 & 0 & 0 \\
    1 & 1 & 1 & 1 & 1 & 1 
  \end{tabular}
\end{center}

$f$ und $g$ sind äquivalent.

\subex{Algebraischer Äquivalenzbeweis}


\subex{Schaltung}
\begin{center}
  \begin{circuitikz}
  % start nodes
  \node[anchor=east] at(4, 3) (b) {B};
  \node[anchor=east] at(4, 2) (a) {A};
  \node[anchor=east] at(4, 1) (c) {C};
  \node[anchor=east] at(4, 0) (d) {D};

  % ~B & A
  \node[and port] at(7, 2.725) (not B and A) {};
  \draw ($(b) + (0.2, 0)$) to[short, -o] ($(not B and A.in 1) + (0.35, 0)$) {};
  \draw ($(a) + (0.2, 0)$) to[short, -*] (5, 2) to (5, 2.45) to (not B and A.in 2) {};

  % C & ~A
  \node[and port] at(7, 1.275) (C and not A) {};
  \draw ($(c) + (0.2, 0)$) to (C and not A.in 2) {};
  \draw (5, 2) to (5, 1.55) to[short, -o] ($(C and not A.in 1) + (0.35, 0)$) {};

  % A & B & C & D
  \node[rectangle, draw, minimum width=1.2cm, minimum height=3.7cm, thick, anchor=center] at(2, 1.5) (bigAnd) {\rmfamily \&};
  \draw ($(b) - (0.2, 0)$) to (2.6, 3);
  \draw ($(a) - (0.2, 0)$) to (2.6, 2);
  \draw ($(c) - (0.2, 0)$) to (2.6, 1);
  \draw ($(d) - (0.2, 0)$) to (2.6, 0);
  \draw (1.4, 1.5) to (1, 1.5) {};

  % combination of all outputs
  \draw (1, 1.5) to (1, -0.75) to (7.5, -0.75) to[short, -*] (7.5, 1.275) to (C and not A.out);
  \draw (not B and A.out) to ($(not B and A.out) + (0.3, 0)$) to (7.5, 1.275) to (9, 1.275);

  % dummy to avoid underful \vbox
  \node at(0,-1) (dummy) {};
  \end{circuitikz}
\end{center}


\ex{Sicherheitstür}{4 + 8 + 6 = 18}

\subex{Veriablen}

\subex{Schaltfunktion}

\subex{Sheffer stroke}


\addex{Bignum}



\end{document}
