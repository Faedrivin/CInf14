\documentclass{CInf_practice}

\sheet{2}{Beispiel-CPU H6809 und Schaltfunktionen}

\begin{document}

\cinftitle

\ex{Division mit Quotient und Rest}{6 + 12 + 6 = 24}
\begin{assemblertable}
% Adr & Op & Oper & Label  & Opc  & Oper   & Comment                          \\
      &    &      &        &      &        &                                  \\
      &    &      &        &      &        &                                  \\
      &    &      &        &      &        &                                  \\
      &    &      &        &      &        &                                  \\
      &    &      &        &      &        &                                  \\
      &    &      &        &      &        &                                  \\
      &    &      &        &      &        &                                  \\
      &    &      &        &      &        &                                  \\
      &    &      &        &      &        &                                  \\
      &    &      &        &      &        &                                  \\
      &    &      &        &      &        &                                  \\
      &    &      &        &      &        &                                  \\
      &    &      &        &      &        &                                  \\
      &    &      &        &      &        &                                  \\
      &    &      &        &      &        &                                  \\
      &    &      &        &      &        &                                  \\
      &    &      &        &      &        &                                  \\
      &    &      &        &      &        &                                  \\
      &    &      &        &      &        &                                  \\
      &    &      &        &      &        &                                  \\
\end{assemblertable}


\ex{Addition von BCD-Zahlen}{6 + 12 = 18}


\ex{Absorptionsgesetze}{7 + 7 = 14}
\subex{}
\begin{align*}
a\cdot \left( a + b \right) &= a & \\
todo
\end{align*}

\subex{}
\begin{align*}% not sure if we are allowed to use associative law!
      \comp a + \left(a        + b\right) &= 1 &\\ 
\left(\comp a +       a\right) + b        &= 1 &\text{(Assoziativgesetz)}\\ 
              1                +        b &= 1 &\text{((4): $a+\comp a = 1$)} \\
              1                           &= 1 &\text{(Neutr. Element)}\\
                                          &    &\square
\end{align*}

\ex{Schaltfunktionen}{6 + 14 + 6 = 26}

\subex{Wertetabelle $f$ und $g$}

\begin{center}
  \begin{tabular}{cccc|c|c}
    \bf a & \bf b & \bf c & \bf d & \bf f & \bf g \\ \hline
    0 & 0 & 0 & 0 & 0 & 0 \\
    0 & 0 & 0 & 1 & 0 & 0 \\
    0 & 0 & 1 & 0 & 1 & 1 \\
    0 & 0 & 1 & 1 & 1 & 1 \\
    0 & 1 & 0 & 0 & 0 & 0 \\
    0 & 1 & 0 & 1 & 0 & 0 \\
    0 & 1 & 1 & 0 & 1 & 1 \\
    0 & 1 & 1 & 1 & 1 & 1 \\
    1 & 0 & 0 & 0 & 1 & 1 \\
    1 & 0 & 0 & 1 & 1 & 1 \\
    1 & 0 & 1 & 0 & 1 & 1 \\
    1 & 0 & 1 & 1 & 1 & 1 \\
    1 & 1 & 0 & 0 & 0 & 0 \\
    1 & 1 & 0 & 1 & 0 & 0 \\
    1 & 1 & 1 & 0 & 0 & 0 \\
    1 & 1 & 1 & 1 & 1 & 1 
  \end{tabular}
\end{center}

$f$ und $g$ sind äquivalent.

\subex{Algebraischer Äquivalenzbeweis}\allowdisplaybreaks
\begin{align*}
f &= \comp{ab + \comp a \comp c} &&+ \comp{ \comp{ bcd + c \comp d } + \comp{acd + a \comp b d} } &  \\
  &= \comp{ab} \cdot \comp{\comp a \comp c} &&+ \comp{ \comp{ bcd + c \comp d } } \cdot \comp{ \comp{acd + a \comp b d} } & \text{(De Morgan)} \\  
  &= \left(\comp a + \comp b \right) \cdot \left( \comp {\comp a} + \comp {\comp c} \right) &&+ \comp{ \comp{ bcd + c \comp d } } \cdot \comp{ \comp{acd + a \comp b d} } & \text{(De Morgan)} \\
  &= \left(\comp a + \comp b \right) \cdot \left( a + c \right)&&+ \left( bcd + c \comp d \right) \cdot \left( acd + a \comp b d \right) & \text{(Eind. d. Kompl.)} \\
  &= \comp a a + \comp a c + \comp b a + \comp b c &&+ bcdacd + bcda\comp b d + c\comp d acd + c\comp d a \comp b d & \text{(Distrib.)} \\
  &= \comp a a + \comp a c + \comp b a + \comp b c &&+ abccdd + ab\comp b cdd + accd\comp d + a\comp b c d\comp d & \text{(Komm.)} \\
  &= 0 + \comp a c + \comp b a + \comp b c &&+ abccdd + a 0 cdd + acc0 + a\comp b c 0 & \text{((4): $a\comp a = 0$)} \\
  &= 0 + \comp a c + \comp b a + \comp b c &&+ abccdd + 0 + 0 + 0 & \text{(Neutr. Elem.)} \\
  &= \comp a c + \comp b a + \comp b c &&+ abccdd & \text{(Nullelement)} \\
  &= \comp a c + \comp b a + \comp b c &&+ abcd & \text{(Idempotenz)} \\
  &= \comp a c + \comp b a + \comp b c + \comp b c &&+ abcd & \text{(Idempotenz)} \\
  &= \comp a c + \comp b a + \comp b c + \comp b c \cdot 1 &&+ abcd & \text{(Einselement)} \\
  &= \comp a c + \comp b a + \comp b c + \comp b c \cdot \left(a+\comp a\right) &&+ abcd & \text{((4): $a+\comp a = 1$)} \\
  &= \comp a c + \comp b a + \comp b c + \comp b c a + \comp b c \comp a && + abcd & \text{(Distrib.)} \\
  &= \comp a c + \comp b a + \comp b c + \comp b c + \comp b c a + \comp b c \comp a && + abcd & \text{(Idempotenz)} \\
  &= \comp a cc + \comp b \comp b a + \comp b \comp b c + \comp b cc + \comp b c a + \comp b c \comp a && + abcd & \text{(Idempotenz)} \\
  &= \left( c + \comp b \right)\left( a\comp b + \comp a c + \comp b c\right) && + abcd & \text{(Distrib.)} \\
  &= \left( c + \comp b \right)\left( 0 + \comp a c + \comp b c\right) && + abcd & \text{(Nullelement)} \\
  &= \left( c + \comp b \right)\left( a\comp a + a\comp b + \comp a c + \comp b c\right) && + abcd & \text{((4): $a\comp a = 0$)} \\
  &= \left( c + \comp b \right)\left( a\left(\comp a + \comp b\right)+c\left(\comp a + \comp b\right) \right) && + abcd & \text{(Distrib.)} \\
  &= \left( c + \comp b \right)\left( \left(a+c\right)\left(\comp a + \comp b\right)\right) && + abcd & \text{(Distrib.)} \\
  &= \left( c + \comp b \right)\left(a+c\right)\left(\comp a + \comp b\right) && + abcd & \text{(Distrib.)} \\
  &= \comp{\comp c b} \cdot \comp{\comp a \comp c} \cdot \comp{ab} && + abcd & \text{(De Morgan)} \\
  &= \comp{\comp c b +\comp a \comp c} \cdot \comp{ab} && + abcd & \text{(De Morgan)} \\
  &= \comp{\comp c b +\comp a \comp c + ab} && + abcd & \text{(De Morgan)} \\
  &= \comp{0 + \comp c b +\comp a \comp c + ab} && + abcd & \text{(Nullelement)} \\
  &= \comp{a\comp a + \comp c b +\comp a \comp c + ab} && + abcd & \text{((4): $a\comp a=0$)} \\
  &= \comp{a\left(\comp a+ b\right) + \comp c \left(b +\comp a\right)} && + abcd & \text{(Distrib.)} \\
  &= \comp{a\left(\comp a+ b\right) + \comp c \left(\comp a + b\right)} && + abcd & \text{(Komm.)} \\
  &= \comp{\left(\comp a+ b\right) \left(a + \comp c \right)} && + abcd & \text{(Distrib.)} \\
  &= \comp{ \comp a+ b } + \comp{a + \comp c} && + abcd & \text{(De Morgan)} \\
  &= a \comp b + \comp a c && + abcd & \text{(De Morgan)} \\
  &= \comp a c + a \comp b && + abcd & \text{(Komm.)} \\
  &= g && &\square
\end{align*}


\subex{Schaltung}
\begin{center}
  \begin{circuitikz}
  % start nodes
  \node[anchor=east] at(4, 3) (b) {B};
  \node[anchor=east] at(4, 2) (a) {A};
  \node[anchor=east] at(4, 1) (c) {C};
  \node[anchor=east] at(4, 0) (d) {D};

  % ~B & A
  \node[and port] at(7, 2.725) (not B and A) {};
  \draw ($(b) + (0.2, 0)$) to[short, -o] ($(not B and A.in 1) + (0.35, 0)$);
  \draw ($(a) + (0.2, 0)$) to[short, -*] (5, 2) 
                           to (5, 2.45) 
                           to (not B and A.in 2);

  % C & ~A
  \node[and port] at(7, 1.275) (C and not A) {};
  \draw ($(c) + (0.2, 0)$) to (C and not A.in 2) {};
  \draw (5, 2) to (5, 1.55) 
               to[short, -o] ($(C and not A.in 1) + (0.35, 0)$) {};

  % A & B & C & D
  \node[rectangle, draw, minimum width=1.2cm, 
        minimum height=3.7cm, thick, anchor=center] at(2, 1.5) (bigAnd) 
        {\rmfamily \&};
    % 4 in, 1 out
  \draw ($(b) - (0.2, 0)$) to (2.6, 3);
  \draw ($(a) - (0.2, 0)$) to (2.6, 2);
  \draw ($(c) - (0.2, 0)$) to (2.6, 1);
  \draw ($(d) - (0.2, 0)$) to (2.6, 0);
  \draw (1.4, 1.5) to (1, 1.5) {};

  % combination of all outputs
  \draw (1, 1.5) to (1, -0.75) 
                 to (7.5, -0.75) 
                 to[short, -*] (7.5, 1.275) 
                 to (C and not A.out);
  \draw (not B and A.out) to ($(not B and A.out) + (0.3, 0)$) 
                          to (7.5, 1.275) 
                          to (9, 1.275);

  % dummy to avoid underful \vbox
  \node at(0,-1) (dummy) {};
  \end{circuitikz}
\end{center}


\ex{Sicherheitstür}{4 + 8 + 6 = 18}

\subex{Variablen}
Als Variablen führen wir ein $K_1$, $K_2$, $B$ und $F$. $K_1$ und $K_2$ 
stehen für die Schlüssel 1 und 2 und sind 1, während der jeweils richtige 
Schlüssel im Schloss gedreht wird, sonst 0. $B$ steht für den Knopf und ist 1, 
während der Knopf gedrückt ist. $F$ steht für den Fingerabdruck und ist 1, 
während der korrekte Finger auf die Sensorfläche gelegt ist.

\subex{Schaltfunktion}
Die Schaltfunktion $t$ kann durch einfaches Kombinieren erstellt werden: Für 
Methode 1 müssen Schlüssel 1 und Schlüssel 2 gleichzeitig gedreht sein, das 
heißt $K_1K_2$. Da zusätzlich der Knopf gedrückt werden muss, erweitert sich 
der Ausdruck auf $K_1K_2B$. Die zweite Methode erlaubt den Fingerabdruck und 
den Knopf, also $FB$. Da beide Methoden individuell zulässig sind, können die 
beiden Formel ODER-verknüpft werden: $K_1K_2B+FB$. Durch das 
Distributivgesetz kann die Formel noch vereinfacht werden:

\begin{align*}
t = K_1K_2B+FB = (K_1K_2+F)B
\end{align*}

\subex{Sheffer stroke}
{ % group for some special commands - note I used AND and ANDouter because
  % I found no easy way to scale parentheses automatically for nested
  % commands.
\tiny
\def\ShefferAND#1#2{\big(\left(#1\middle|#2\right)\big|\left(#1\middle|#2\right)\big)}
\def\ShefferOR#1#2{\bigg(\Big(#1\Big|#1\Big)\bigg|\Big(#2\Big|#2\Big)\bigg)}
\def\ShefferANDouter#1#2{\Bigg(#1\Bigg|#2\Bigg)\Bigg|\Bigg(#1\Bigg|#2\Bigg)}
\begin{align*}
% the first three lines are just to show the expansion steps
& && (K_1K_2+F)B\\
&=&& \bigg(\ShefferAND{K_1}{K_2}+F\bigg)B\\
&=&& \ShefferOR{\ShefferAND{K_1}{K_2}}{F}B\\
&=&& \ShefferANDouter{\ShefferOR{\ShefferAND{K_1}{K_2}}{F}}{B}
\end{align*}
} % end group




\addex{Bignum}



\end{document}
