\documentclass{CInf_practice}

\sheet{2}{Beispiel-CPU H6809 und Schaltfunktionen}

\begin{document}

\cinftitle

\ex{Division mit Quotient und Rest}{6 + 12 + 6 = 24}

\subex{Programmablaufplan}
\begin{center}
  \begin{tikzpicture}
    \def\a{1cm}
    \def\b{2cm}
    \node[cloud]                              (begin)     {Begin};
    \node[block,    below=.5cm of begin]      (init)      {A \la Z$_2$};
    \node[decision, below=1cm of init]        (divzero)   {A==0?};
    \node[block,    left =2cm of divzero]     (preloop)   {REST \la Z$_1$};
    \node[cloud,    below=1cm of preloop]     (loopOuter) {LoopO};
    \node[block,    below=1cm of loopOuter]   (initLoop)  {\begin{tabular}{l}
                                                             A\la REST\\
                                                             X\la Z$_2$
                                                           \end{tabular}};
    \node[cloud,    below=1cm of initLoop]    (loopInner) {LoopI};
    \node[decision, right=3cm of loopInner]   (xzero)     {X==0?};
    \node[decision, right=2.5cm of xzero]     (azero)     {A==0?};
    \node[block,    below=.5cm of azero]      (loopContentInner) {\begin{tabular}{l}
                                                                    X\la X-1\\
                                                                    A\la A-1
                                                                  \end{tabular}};
    % split?
    \node[block,    right=2.3cm of loopOuter] (loopContentOuter) {\begin{tabular}{l}
                                                                    REST \la A\\
                                                                    ERG \la ERG+1
                                                                  \end{tabular}};
    \node[block,    right=2cm of divzero]   (error) {\begin{tabular}{l}
                                                       ERG\la \$00\\
                                                       REST\la \$FF
                                                     \end{tabular}};
    \node[cloud,    below=1cm of error]     (exit) {Exit};
    
    \path[line] (begin)            -- (init);
    \path[line] (init)             -- (divzero);
    \path[line] (divzero)          -- node[anchor=south]{n} (preloop);preloop
    \path[line] (divzero)          -- node[anchor=south]{y} (error);
    \path[line] (error)            -- (exit);
    \path[line] (preloop)          -- (loopOuter);
    \path[line] (loopOuter)        -- (initLoop);
    \path[line] (initLoop)         -- (loopInner);
    \path[line] (loopInner)        -- (xzero);
    \path[line] (xzero)            -- node[anchor=south]{n} (azero);
    \path[line] (azero)            -- node[anchor=east]{n}  (loopContentInner);
    \path[line] (loopContentInner) -| (loopInner);
    \path[line] (xzero)            -- node[anchor=east]{y}  (loopContentOuter);
    \path[line] (loopContentOuter) -- (loopOuter);
    \path[line] (azero)            -- node[anchor=east]{y}  (exit);
  \end{tikzpicture}
\end{center}

\subex{Assemblersequenz}
\begin{assemblertable}
% Adr & Op & Oper & Label  & Opc  & Oper     & Comment                        \\
      &    &      &        & .ORG & \$0500   & data space                     \\\hline
 0500 &    &      & Z1     & .BYTE& 1        &                                \\\hline
 0501 &    &      & Z2     & .BYTE& 1        &                                \\\hline
 0502 &    &      & ERG    & .BYTE& 1        &                                \\\hline
 0503 &    &      & REST   & .BYTE& 1        &                                \\\hline
      &    &      &        & .ORG & \$0600   & program space                  \\\hline% cycles
 0600 & B6 & 0501 & BEGIN: & LDA  & Z2       & load divisor                   \\\hline% 4
 0603 & 26 & 0A   &        & BNE  & LOOPS    & if divisor is not zero: divide \\\hline% 2
 0605 & B7 & 0502 &        & STA  & ERG      & store A (zero) to ERG          \\\hline% 4
 0608 & 43 &      &        & COMA &          & flip it                        \\\hline% 1
 0609 & B7 & 0503 &        & STA  & REST     & store A to REST                \\\hline% 4
 060C & 7E & 1000 &        & JMP  & \$1000   & exit program                   \\\hline% 3
 060F & B6 & 0500 & LOOPS: & LDA  & Z1       & load dividend                  \\\hline% 4
 0612 & B7 & 0503 &        & STA  & REST     & dividend is initial rest       \\\hline% 4
 0615 & B6 & 0503 & LOOPO: & LDA  & REST     & load rest                      \\\hline% 4
 0618 & 8E & 0501 &        & LDX  & Z2       & load divisor                   \\\hline% 3
 061B & 27 & 0B   & LOOPI: & BEQ  & STRST    & if X is zero, divisor fit      \\\hline% 2
 061D & 81 & 00   &        & CMPA & \#00     & update zero flag               \\\hline% 2
 061F & 27 & 14   &        & BEQ  & EXIT     & if A is zero, it is done       \\\hline% 2
 0621 & 8B & FF   &        & ADDA & \#\$FF   & decrement A                    \\\hline% 2
 0623 & 30 & FFFF &        & ADDX & \#\$FFFF & decrement X                    \\\hline% 3
 0626 & 20 & F3   &        & BRA  & LOOPI    & go back and check X again      \\\hline% 2
 0628 & B7 & 0503 & STRST: & STA  & REST     & store rest                     \\\hline% 3
 062B & B6 & 0502 &        & LDA  & ERG      & load result                    \\\hline% 3
 062E & 8B & 01   &        & ADDA & \#\$01   & increment A                    \\\hline% 2
 0630 & B7 & 0502 &        & STA  & ERG      & store new result               \\\hline% 4
 0633 & 20 & E0   &        & BRA  & LOOPO    & continue outer loop            \\\hline% 2
 0635 & 7E & 1000 & EXIT:  & JMP  & \$1000   & exit program                   \\\hline% 3
\end{assemblertable}

\subex{Befehlszahl}
Der Minimalfall ist die Division einer beliebigen Zahl mit 0. Hierbei werden 
sechs Befehle in $18$ Takten abgearbeitet.
Der Maximalfall ist die Division von 255 mit 1. Hierzu werden $8+14\cdot 255+6
=3584$ Befehle (Takte: $14+36\cdot 255+9=9203$) (Initialisierung, Loop, Beenden)
benötigt. 
% ~2.6 Takte pro Befehl, scheint etwa zu passen. Bei Langeweile darf aber nachgerechnet werden.

\ex{Addition von BCD-Zahlen}{6 + 12 = 18}

\subex{Programmablaufplan}
\tikzset{block/.append style={minimum height=1.7em,text width=6em}}
\begin{tikzpicture}[node distance=.3cm]
   \node[cloud] (begin) {BEGIN};
   \node[block,below=1cm of begin] (A<-Z1) {A \la Z1};
   \node[block,below=of A<-Z1] (A<-A+Z2) {A \la A + Z2};
   \node[block,below=of A<-A+Z2] (ERG<-A) {ERG \la A};
   \node[cloud, right=1cm of ERG<-A] (fix_lower) {FIX\_LOWER};
   \node[block,above=1cm of fix_lower] (A<-A+06) {A \la A + 06$_{16}$};
   \node[block,above=of A<-A+06] (Aand10) {A \&= 10$_{16}$};
   \node[decision,above right=of A<-A+06] (aIs0?) {A == 0?};
   \node[block,below=of aIs0?] (A<-ERG) {A \la ERG};
   \node[block,below=of A<-ERG] (A<-A+06-3) {A \la A + 06$_{16}$};
   \node[block,below=of A<-A+06-3] (ERG<-A-2) {ERG \la A};
   \node[cloud,right=of aIs0?] (fix_upper) {FIX\_UPPER};
   \node[block,below=of fix_upper] (A>>4) {A >> 4};
   \node[block,below=of A>>4] (Aand15) {A \&= 0F$_{16}$};
   \node[block,below=of Aand15] (A<-A+06-2) {A \la A + 06$_{16}$};
   \node[block,below=of A<-A+06-2] (Aand10-2) {A \&= 10$_{16}$};
   \node[decision,below=.5cm of Aand10-2] (aIs0?-2) {A == 0?};
   \node[block,left=of aIs0?-2] (A<-ERG-2) {A \la ERG};
   \node[block,left=of A<-ERG-2] (A<-A+60) {A \la A + 60$_{16}$};
   \node[block,left=of A<-A+60] (ERG<-A-3) {ERG \la A};
   \node[cloud,left=of ERG<-A-3] (end) {END};

   \path[line] (begin) -- (A<-Z1);
   \path[line] (A<-Z1) -- (A<-A+Z2);
   \path[line] (A<-A+Z2) -- (ERG<-A);
   \path[line] (ERG<-A) -- (fix_lower);
   \path[line](fix_lower) -- (A<-A+06);
   \path[line](A<-A+06) -- (Aand10);
   \path[line] (Aand10) |- (aIs0?);
   \path[line] (aIs0?) -- node[left] {NO} (A<-ERG);
   \path[line] (A<-ERG) -- (A<-A+06-3);
   \path[line] (A<-A+06-3) -- (ERG<-A-2);
   \path[draw] (ERG<-A-2) -| ($(fix_upper.west) + (-1.2cm,0cm)$) -- (fix_upper);
   \path[line] (aIs0?) -- node[above] {YES} (fix_upper);
   \path[line] (fix_upper) -- (A>>4);
   \path[line] (A>>4) -- (Aand15);
   \path[line] (Aand15) -- (A<-A+06-2);
   \path[line] (A<-A+06-2) -- (Aand10-2);
   \path[line] (Aand10-2) -- (aIs0?-2);
   \path[line] (aIs0?-2) -- node[above] {NO} (A<-ERG-2);
   \path[line] (A<-ERG-2) -- (A<-A+60);
   \path[line] (A<-A+60) -- (ERG<-A-3);
   \path[line] (ERG<-A-3) -- (end);
   \path[line] (aIs0?-2) |- ($(end.south) + (0cm,-1cm)$) -- (end);
\end{tikzpicture}

\subex{Assemblerprogramm}

\begin{assemblertable}
   &  &  & BEGIN: & LDA    & Z1  & First regular binary addition \\\hline
   &  &  &        & ADDA   & Z2  & \\\hline
   &  &  &        & STA    & ERG & \\\hline
   &  &  & FIX\_LOWER: & ADDA & \#0x06 & +6 is equiv. to -10 \\\hline
   &  &  &            & ANDA & \#0x10 & Check if carry was generated \\\hline
   &  &  &            & BEQ  & FIX\_UPPER & If not, do nothing, move to high nibble \\\hline
   &  &  &            & LDA  & ERG & Otherwise, roll back and do the computation again\\\hline
   &  &  &            & ADDA  & \#0x06 & \\\hline
   &  &  &            & STA  & ERG & This time, save it\\\hline
   &  &  & FIX\_UPPER: & RORA & &For the upper nibble, shift to right \\\hline
   &  &  &             & RORA & &\\\hline
   &  &  &             & RORA & &\\\hline
   &  &  &             & RORA & &\\\hline
   &  &  &             & ANDA & \#0x0F & Mask out high nibble\\\hline
   &  &  &             & ADDA & \#0x06 & Again try to substract 10\\\hline
   &  &  &             & ANDA & \#0x10 & Check if correct\\\hline
   &  &  &             & BEQ & END & If not, don't save and exit\\\hline
   &  &  &             & LDA & ERG & Else do again and save it\\\hline
   &  &  &            & ADDA  & \#0x60 & This time, add to high nibble, leave low nibble alone\\\hline
   &  &  &            & STA  & ERG & \\\hline
   &  &  & END:       & JMP  & \$1000 & \\\hline
\end{assemblertable}

\ex{Absorptionsgesetze}{7 + 7 = 14}
Wir beweisen vorab die Idempotenzgesetze.
\begin{IEEEeqnarray*}{rC?l}
   a + 0 & = a & \text{(Nullelement)} \\
   a + (a \cdot \comp a) & = a & \text{(Komplementäres Element)} \\
   (a + a) \cdot (a + \comp a) & = a & \text{(Distributivgesetz)} \\
   (a + a) \cdot 1 & = a & \text{(Komplementäres Element)} \\
   a + a & = a & \text{(Einselement)}
\end{IEEEeqnarray*}
\hfill{$\square$}

\begin{IEEEeqnarray*}{rC?l}
   a \cdot 1 & = a &  \\
   a \cdot (a + \comp a) & = a & \text{(Komplementäres Element)} \\
   (a \cdot a) + (a \cdot \comp a) & = a & \text{(Distributivgesetz)} \\
   (a \cdot a) + 0 & = a & \text{(Komplementäres Element)} \\
   a \cdot a & = a & \text{(Nullelement)}
\end{IEEEeqnarray*}
\hfill{$\square$}

Außerdem beweisen wir die Existenz der neutralen Elemente.
\begin{IEEEeqnarray*}{rC?l}
          1 + a & = 1 & \\
a + \comp a + a & = 1 & \text{((4): $a+\comp a = 1$)} \\
a + a + \comp a & = 1 & \text{(Kommutativgesetz)} \\
    a + \comp a & = 1 & \text{(Idempotenzgesetz, s.o.)} \\
              1 & = 1 & \text{((4): $a+\comp a = 1$)} \\
\end{IEEEeqnarray*}
\hfill{$\square$}

\begin{IEEEeqnarray*}{rC?l}
          0 * a & = 0 & \\
a * \comp a * a & = 0 & \text{((4): $a\cdot\comp a = 0$)} \\
a * a * \comp a & = 0 & \text{(Kommutativgesetz)} \\
    a * \comp a & = 0 & \text{(Idempotenzgesetz, s.o.)} \\
              0 & = 0 & \text{((4): $a\cdot\comp a = 0$)} \\
\end{IEEEeqnarray*}
\hfill{$\square$}

\subex{}
\begin{IEEEeqnarray*}{rC?l}
   a\cdot \left( a + b \right) &= a & \hspace{1cm}\\
   (b + \comp b) \cdot a \cdot (a + b) & = a & \text{(Neutrales Element)}\\
   ((b \cdot a) + (\comp b \cdot a)) \cdot (a + b) & = a &\text{(Distributivgesetz)}\\
   ((b \cdot a) \cdot (a + b)) + ((\comp b \cdot a) \cdot (a + b)) & =  a & \text{(Distributivgesetz)}\\
   (((a + b) \cdot b) \cdot ((a + b) \cdot a)) + ((\comp b \cdot a) \cdot (a + b))& =  a & \text{(Distributivgesetz)}\\
   (((a \cdot b) + (b \cdot b)) \cdot ((a \cdot b) + (a\cdot a))) + ((\comp b \cdot a) \cdot (a + b))& =  a &  \text{(Distributivgesetz)}\\
   ((a \cdot b) + (a \cdot b)) + ((\comp b \cdot a) \cdot (a + b))& =  a & \text{(Distributivgesetz, invers)}\\
   (a \cdot b) + ((\comp b \cdot a) \cdot (a + b))& =  a & \text{(Idempotenzgesetz)}\\
   (a \cdot b) + ((\comp b \cdot a \cdot a) + (\comp b \cdot a \cdot b))& =  a & \text{(Distributivgesetz)}\\
   (a \cdot b) + ((\comp b \cdot a) + 0)& =  a & \text{($a \cdot \comp a = 0$)}\\
   (a \cdot b) + (\comp b \cdot a)& =  a & \text{($a + 0 = a$)}\\
   a \cdot (b + \comp b) & = a & \text{(Distributivgesetz, invers)} \\
   a & = a & \text{($b + \comp b = 1$)}\\
\end{IEEEeqnarray*}
\hfill{$\square$}

\subex{}
\begin{IEEEeqnarray*}{rC?l} % not sure if we are allowed to use associative law!
   \comp a + \left(a        + b\right) &= 1 &\\ 
   \left(\comp a +       a\right) + b        &= 1 &\text{(Assoziativgesetz)} \\ 
   1                +        b &= 1 &\text{((4): $a+\comp a = 1$)} \\
   1                           &= 1 &\text{(Neutrales Element)} \\
\end{IEEEeqnarray*}
\hfill{$\square$}

\ex{Schaltfunktionen}{6 + 14 + 6 = 26}

\subex{Wertetabelle $f$ und $g$}

\begin{center}
  \begin{tabular}{cccc|c|c}
    \bf a & \bf b & \bf c & \bf d & \bf f & \bf g \\ \hline
    0 & 0 & 0 & 0 & 0 & 0 \\
    0 & 0 & 0 & 1 & 0 & 0 \\
    0 & 0 & 1 & 0 & 1 & 1 \\
    0 & 0 & 1 & 1 & 1 & 1 \\
    0 & 1 & 0 & 0 & 0 & 0 \\
    0 & 1 & 0 & 1 & 0 & 0 \\
    0 & 1 & 1 & 0 & 1 & 1 \\
    0 & 1 & 1 & 1 & 1 & 1 \\
    1 & 0 & 0 & 0 & 1 & 1 \\
    1 & 0 & 0 & 1 & 1 & 1 \\
    1 & 0 & 1 & 0 & 1 & 1 \\
    1 & 0 & 1 & 1 & 1 & 1 \\
    1 & 1 & 0 & 0 & 0 & 0 \\
    1 & 1 & 0 & 1 & 0 & 0 \\
    1 & 1 & 1 & 0 & 0 & 0 \\
    1 & 1 & 1 & 1 & 1 & 1 
  \end{tabular}
\end{center}

$f$ und $g$ sind äquivalent.

% according to practice session this should be possible in about five lines. 
% I don't see how. But I'm certain this is not optimal.
\subex{Algebraischer Äquivalenzbeweis}\allowdisplaybreaks
\begin{align*}
f &= \comp{ab + \comp a \comp c} &&+ \comp{ \comp{ bcd + c \comp d } + \comp{acd + a \comp b d} } &  \\
  &= \comp{ab} \cdot \comp{\comp a \comp c} &&+ \comp{ \comp{ bcd + c \comp d } } \cdot \comp{ \comp{acd + a \comp b d} } & \text{(De Morgan)} \\  
  &= \left(\comp a + \comp b \right) \cdot \left( \comp {\comp a} + \comp {\comp c} \right) &&+ \comp{ \comp{ bcd + c \comp d } } \cdot \comp{ \comp{acd + a \comp b d} } & \text{(De Morgan)} \\
  &= \left(\comp a + \comp b \right) \cdot \left( a + c \right)&&+ \left( bcd + c \comp d \right) \cdot \left( acd + a \comp b d \right) & \text{(Eind. d. Kompl.)} \\
  &= \comp a a + \comp a c + \comp b a + \comp b c &&+ bcdacd + bcda\comp b d + c\comp d acd + c\comp d a \comp b d & \text{(Distrib.)} \\
  &= \comp a a + \comp a c + \comp b a + \comp b c &&+ abccdd + ab\comp b cdd + accd\comp d + a\comp b c d\comp d & \text{(Komm.)} \\
  &= 0 + \comp a c + \comp b a + \comp b c &&+ abccdd + a 0 cdd + acc0 + a\comp b c 0 & \text{((4): $a\comp a = 0$)} \\
  &= 0 + \comp a c + \comp b a + \comp b c &&+ abccdd + 0 + 0 + 0 & \text{(Neutr. Elem.)} \\
  &= \comp a c + \comp b a + \comp b c &&+ abccdd & \text{(Nullelement)} \\
  &= \comp a c + \comp b a + \comp b c &&+ abcd & \text{(Idempotenz)} \\
  &= \comp a c + \comp b a + \comp b c + \comp b c &&+ abcd & \text{(Idempotenz)} \\
  &= \comp a c + \comp b a + \comp b c + \comp b c \cdot 1 &&+ abcd & \text{(Einselement)} \\
  &= \comp a c + \comp b a + \comp b c + \comp b c \cdot \left(a+\comp a\right) &&+ abcd & \text{((4): $a+\comp a = 1$)} \\
  &= \comp a c + \comp b a + \comp b c + \comp b c a + \comp b c \comp a && + abcd & \text{(Distrib.)} \\
  &= \comp a c + \comp b a + \comp b c + \comp b c + \comp b c a + \comp b c \comp a && + abcd & \text{(Idempotenz)} \\
  &= \comp a cc + \comp b \comp b a + \comp b \comp b c + \comp b cc + \comp b c a + \comp b c \comp a && + abcd & \text{(Idempotenz)} \\
  &= \left( c + \comp b \right)\left( a\comp b + \comp a c + \comp b c\right) && + abcd & \text{(Distrib.)} \\
  &= \left( c + \comp b \right)\left( 0 + \comp a c + \comp b c\right) && + abcd & \text{(Nullelement)} \\
  &= \left( c + \comp b \right)\left( a\comp a + a\comp b + \comp a c + \comp b c\right) && + abcd & \text{((4): $a\comp a = 0$)} \\
  &= \left( c + \comp b \right)\left( a\left(\comp a + \comp b\right)+c\left(\comp a + \comp b\right) \right) && + abcd & \text{(Distrib.)} \\
  &= \left( c + \comp b \right)\left( \left(a+c\right)\left(\comp a + \comp b\right)\right) && + abcd & \text{(Distrib.)} \\
  &= \left( c + \comp b \right)\left(a+c\right)\left(\comp a + \comp b\right) && + abcd & \text{(Distrib.)} \\
  &= \comp{\comp c b} \cdot \comp{\comp a \comp c} \cdot \comp{ab} && + abcd & \text{(De Morgan)} \\
  &= \comp{\comp c b +\comp a \comp c} \cdot \comp{ab} && + abcd & \text{(De Morgan)} \\
  &= \comp{\comp c b +\comp a \comp c + ab} && + abcd & \text{(De Morgan)} \\
  &= \comp{0 + \comp c b +\comp a \comp c + ab} && + abcd & \text{(Nullelement)} \\
  &= \comp{a\comp a + \comp c b +\comp a \comp c + ab} && + abcd & \text{((4): $a\comp a=0$)} \\
  &= \comp{a\left(\comp a+ b\right) + \comp c \left(b +\comp a\right)} && + abcd & \text{(Distrib.)} \\
  &= \comp{a\left(\comp a+ b\right) + \comp c \left(\comp a + b\right)} && + abcd & \text{(Komm.)} \\
  &= \comp{\left(\comp a+ b\right) \left(a + \comp c \right)} && + abcd & \text{(Distrib.)} \\
  &= \comp{ \comp a+ b } + \comp{a + \comp c} && + abcd & \text{(De Morgan)} \\
  &= a \comp b + \comp a c && + abcd & \text{(De Morgan)} \\
  &= \comp a c + a \comp b && + abcd & \text{(Komm.)} \\
  &= g && &\square
\end{align*}


\subex{Schaltung}
\begin{center}
  \begin{circuitikz}
  % start nodes
  \node[anchor=east] at(4, 3) (b) {B};
  \node[anchor=east] at(4, 2) (a) {A};
  \node[anchor=east] at(4, 1) (c) {C};
  \node[anchor=east] at(4, 0) (d) {D};

  % ~B & A
  \node[and port] at(7, 2.725) (not B and A) {};
  \draw ($(b) + (0.2, 0)$) to[short, -o] ($(not B and A.in 1) + (0.325, 0)$);
  \draw ($(a) + (0.2, 0)$) to[short, -*] (5, 2) 
                           to (5, 2.45) 
                           to (not B and A.in 2);

  % C & ~A
  \node[and port] at(7, 1.275) (C and not A) {};
  \draw ($(c) + (0.2, 0)$) to (C and not A.in 2) {};
  \draw (5, 2) to (5, 1.55) 
               to[short, -o] ($(C and not A.in 1) + (0.325, 0)$) {};

  % A & B & C & D
  \node[rectangle, draw, minimum width=1.2cm, 
        minimum height=3.7cm, thick, anchor=center] at(2, 1.5) (bigAnd) 
        {\rmfamily \&};
    % 4 in, 1 out
  \draw ($(b) - (0.2, 0)$) to (2.6, 3);
  \draw ($(a) - (0.2, 0)$) to (2.6, 2);
  \draw ($(c) - (0.2, 0)$) to (2.6, 1);
  \draw ($(d) - (0.2, 0)$) to (2.6, 0);
  \draw (1.4, 1.5) to (1, 1.5) {};

  % combination of all outputs
  \draw (1, 1.5) to (1, -0.75) 
                 to (7.5, -0.75) 
                 to[short, -*] (7.5, 1.275) 
                 to (C and not A.out);
  \draw (not B and A.out) to ($(not B and A.out) + (0.3, 0)$) 
                          to (7.5, 1.275) 
                          to (9, 1.275);

  % dummy to avoid underful \vbox
  \node at(0,-1) (dummy) {};
  \end{circuitikz}
\end{center}


\ex{Sicherheitstür}{4 + 8 + 6 = 18}

\subex{Variablen}
Als Variablen führen wir ein $K_1$, $K_2$, $B$ und $F$. $K_1$ und $K_2$ 
stehen für die Schlüssel 1 und 2 und sind 1, während der jeweils richtige 
Schlüssel im Schloss gedreht wird, sonst 0. $B$ steht für den Knopf und ist 1, 
während der Knopf gedrückt ist. $F$ steht für den Fingerabdruck und ist 1, 
während der korrekte Finger auf die Sensorfläche gelegt ist.

\subex{Schaltfunktion}
Die Schaltfunktion $t$ kann durch einfaches Kombinieren erstellt werden: Für 
Methode 1 müssen Schlüssel 1 und Schlüssel 2 gleichzeitig gedreht sein, das 
heißt $K_1K_2$. Da zusätzlich der Knopf gedrückt werden muss, erweitert sich 
der Ausdruck auf $K_1K_2B$. Die zweite Methode erlaubt den Fingerabdruck und 
den Knopf, also $FB$. Da beide Methoden individuell zulässig sind, können die 
beiden Formel ODER-verknüpft werden: $K_1K_2B+FB$. Durch das 
Distributivgesetz kann die Formel noch vereinfacht werden:

\begin{align*}
t = K_1K_2B+FB = (K_1K_2+F)B
\end{align*}

\subex{Sheffer stroke}
{ % group for some special commands - note I used AND and ANDouter because
  % I found no easy way to scale parentheses automatically for nested
  % commands.
\tiny
\def\ShefferAND#1#2{\big(\left(#1\middle|#2\right)\big|\left(#1\middle|#2\right)\big)}
\def\ShefferOR#1#2{\bigg(\Big(#1\Big|#1\Big)\bigg|\Big(#2\Big|#2\Big)\bigg)}
\def\ShefferANDouter#1#2{\Bigg(#1\Bigg|#2\Bigg)\Bigg|\Bigg(#1\Bigg|#2\Bigg)}
\begin{align*}
% the first three lines are just to show the expansion steps
& && (K_1K_2+F)B\\
&=&& \bigg(\ShefferAND{K_1}{K_2}+F\bigg)B\\
&=&& \ShefferOR{\ShefferAND{K_1}{K_2}}{F}B\\
&=&& \ShefferANDouter{\ShefferOR{\ShefferAND{K_1}{K_2}}{F}}{B}
\end{align*}
} % end group




\addex{Bignum}



\end{document}
