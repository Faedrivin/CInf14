\documentclass{CInf_practice}

\sheet{0}
\title{Dualzahlen und Codes}

\begin{document}

\cinftitle



\ex{Binär-, Oktal- und Hexadezimalzahlen}{25}

\begin{center}
	\begin{tabular}{|c|c|c|c|}
  \hline \bf Dezimal   & \bf Binär          & \bf Oktal  & \bf Hexadezimal \\
  \hline \sl 141.28125 &     10001101.01001 &     215.22 &      8D.48      \\
  \hline     181.625   & \sl 10110101.101   &     265.5  &      B5.A       \\
  \hline     132.625   &     10000100.101   & \sl 204.5  &      84.A       \\
  \hline     315.75    &    100111011.11    &     473.6  & \sl 13B.C       \\
  \hline
  \end{tabular}
\end{center}



\ex{Binäre Addition \& Subtraktion}{10}

\subex

\begin{lstlisting}
    1101     13
  + 1110   + 14
  ------   ----
   11011     27
\end{lstlisting}


\subex

\begin{lstlisting}
    11001     25
  -  1011   - 19
  -------   ----
    00110      6
\end{lstlisting}




\end{document}