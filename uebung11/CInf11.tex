\documentclass{CInf_practice}

\sheet{11}{ATmega16}

\begin{document}
\cinftitle

\ex{Stackverarbeitung}{1 + 8 + 14}
\subex{Adressierung Arrays}
Am besten kann man für Arrayadressierungen in unserem Anwendungsfall auf dem ATmega16 
\emph{Data Indirect with Post-increment} verwenden.

\subex{Programmablaufplan \& Registerbelegung}
Registerbelegung

{ % scope for these short hands:
\def\val{current\_value}
\def\stack{stack\_size}
\def\ptrl{array\_ptr\_low}
\def\ptrh{array\_ptr\_high}

\def\blockW{5.0cm}
\def\decW{3.5cm}

\begin{ctabular}{l|l|l}
Register & Bennenung & Belegung \\\hline
R24 & \val & Der aktuell betrachtete Wert \\
R25 & \stack & Anzahl Elemente auf dem Stack \\
R26 & \ptrl & Low-Byte d. Array Pointers \\
    &       & Länge Ergebnisarray \\
R27 & \ptrh & High-Byte d. Array Pointers \\
X   & X & R26 und R27 sind X\\
\end{ctabular}

\begin{center}
\begin{tikzpicture}
\node[cloud] (begin) {Begin};
\node[block, below=of begin, text width=\blockW] (init)
    {\ptrl \la low(\#0x200) \\ \ptrh \la high(\#0x200) \\ \stack \la \#0};
\node[block, below=of init, text width=\blockW] (loophead) 
    {\val \la (X+)};
\node[decision, below=of loophead, text width=\decW] (odd)
    {lsb(\val) == 1?};
\node[block, below=of odd, text width=\blockW] (push) 
    {PUSH \val \\ \stack \la \stack + 1};
\node[decision, right=of push, text width=\decW] (255)
    {\val\ == 255?};
\node[decision, right=of odd, above=of 255, text width=\decW] (30)
    {\ptrl\ == 30?};
\node[block, right=of 30, text width=\blockW] (reset)
    {\ptrl \la low(\#0x200) \\ \ptrh \la high(\#0x200)};
\node[decision, above=of reset, text width=\decW] (done)
    {\stack\ == 0?};
\node[block, above=of done, text width=\blockW] (pop)
    {POP \val \\ (X+)\la \val \\ \stack \la \stack - 1};
\node[cloud, left=of done] (end) {End};

\draw[->] (begin) -- (init);
\draw[->] (init) -- (loophead);
\draw[->] (loophead) -- (odd);
\draw[->] (odd) -- node[left] {yes} (push);
\draw[->] (odd) -- node[above] {no} (30);
\draw[->] (push) -- (255);
\draw[->] (255) -- node[left] {no} (30);
\draw[->] (255) -| node[below] {yes} (reset);
\draw[->] (30) -- node[above] {no} (loophead);
\draw[->] (30) -- node[above] {yes} (reset);
\draw[->] (reset) -- (done);
\draw[->] (done) -- node[above] {yes} (end);
\draw[->] (done) -- node[left] {no} (pop);
\draw[->] (pop) -- ($(pop)+(\blockW/1.5,0)$) |- (done);

\end{tikzpicture}
\end{center}
} % end scope for \val, \stack, ...



\ex{Division auf dem ATmega16}{8 + 15}


\ex{Fakultät auf dem ATmega16}{8 + 15}

\ex{Einfacher Rechner}{8 + 8 + 15}


\end{document}
