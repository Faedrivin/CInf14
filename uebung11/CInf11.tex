\documentclass{CInf_practice}

\sheet{11}{ATmega16}

\lstset{language={[avr8avra]Assembler},
        morecomment=[n]{/*}{*/},
        morecomment=[l]{//},
     }

\begin{document}
\cinftitle

Alle Programme dieses Aufgabenblattes nutzen folgende Include-Datei \texttt{init\_header.inc}.
\lstinputlisting{init_header.inc}



\ex{Stackverarbeitung}{1 + 8 + 14}


\subex{Adressierung Arrays}
Am besten kann man für Arrayadressierungen in unserem Anwendungsfall auf dem ATmega16 
\emph{Data Indirect with Post-increment} verwenden.


\subex{Programmablaufplan \& Registerbelegung}
{ % scope for these short hands:
\def\val{current\_value}
\def\stack{stack\_size}
\def\ptrl{XL}
\def\ptrh{XH}

\def\blockW{5.0cm}
\def\decW{3.5cm}

\begin{ctabular}{l|l|l}
Register & Bennenung & Belegung \\\hline
R24 & \val & Der aktuell betrachtete Wert \\
R25 & \stack & Anzahl Elemente auf dem Stack \\
R26 & \ptrl & Low-Byte d. Array Pointers \\
    &       & Länge Ergebnisarray \\
R27 & \ptrh & High-Byte d. Array Pointers \\
X   & X & R26 und R27 sind X\\
\end{ctabular}

\begin{center}
\begin{tikzpicture}
\node[cloud] (begin) {Begin};
\node[block, below=of begin, text width=\blockW] (init)
    {\ptrl\ \la low(\#0x200) \\ \ptrh\ \la high(\#0x200) \\ \stack\ \la \#0};
\node[block, below=of init, text width=\blockW] (loophead) 
    {\val\ \la (X+)};
\node[decision, below=of loophead, text width=\decW, shape aspect=3] (odd)
    {odd(\val)?};
\node[block, below=of odd, text width=\blockW] (push) 
    {PUSH \val \\ \stack\ \la \stack + 1};
\node[decision, below=of push, text width=\decW, shape aspect=2.5] (255)
    {\val\ == 255?};
\node[decision, below=of 255, text width=\decW, shape aspect=2.5] (30)
    {\ptrl\ == 30?};
\node[block, below=of 30, text width=\blockW] (reset)
    {\ptrl\ \la low(\#0x200) \\ \ptrh\ \la high(\#0x200)};
\node[decision, below=of reset, text width=\decW, shape aspect=2.5] (done)
    {\stack\ == 0?};
\node[block, below=of done, text width=\blockW] (pop)
    {POP \val \\ (X+)\ \la \val \\ \stack\ \la \stack - 1};
\node[cloud, left=of done] (end) {End};

\draw[->] (begin) -- (init);
\draw[->] (init) -- (loophead);
\draw[->] (loophead) -- (odd);
\draw[->] (odd) -- node[left] {yes} (push);
\draw[->] (odd) -- node[above] {no} ($(odd)+(\blockW/1.5,0)$) |- (255);
\draw[->] (push) -- (255);
\draw[->] (255) -- node[left] {no} (30);
\draw[->] (255) -- node[below] {yes} ($(255)+(-\blockW/1.5,0)$) |- (reset);
\draw[->] (30) -- node[below] {no} ($(30)+(\blockW,0)$) |- (loophead);
\draw[->] (30) -- node[left] {yes} (reset);
\draw[->] (reset) -- (done);
\draw[->] (done) -- node[above] {yes} (end);
\draw[->] (done) -- node[left] {no} (pop);
\draw[->] (pop) -- ($(pop)+(\blockW/1.5,0)$) |- (done);

\end{tikzpicture}
\end{center}
} % end scope for \val, \stack, ...

\subex{Assemblerprogramm}
\lstinputlisting{array_rev.asm}



\ex{Division auf dem ATmega16}{8 + 15}

\subex{Programmablaufplan}


\subex{Assemblerprogramm}

\lstinputlisting{division.asm}



\ex{Fakultät auf dem ATmega16}{8 + 15}

\subex{Programmablaufplan}

\subex{Assemblerprogramm}

\lstinputlisting{factorial.asm}



\ex{Einfacher Rechner}{8 + 8 + 15}

\subex{\texttt{INIT\_PORTS}}

Das gesamte Programm ist in \autoref{lst:calc} \textcolor{red}{(wtf is the number?)} abgedruckt.

\subex{Programmablaufplan und Registerbelegung}

Um den unterschiedlichen Namen der bereits entwickelten Unterprogramme Rechnung
zu tragen, haben die Register mehrere Namen.

\begin{ctabular}{>{\ttfamily}l>{\ttfamily}l}
    \rmfamily Name & \rmfamily Register \\\hline
   numerator, n, summand1, subtrahend & R16 \\
   denominator,summand2, minuend & R17 \\
   quotient, result & R18 \\
   remainder, tmp, operation & R19
\end{ctabular}

\begin{center}
   \begin{tikzpicture}[block/.append style={text width=4cm}]
      \node[cloud] (main) {Main Program};
      \node[below=of main, block] (initsp1) {SP(15..8) \la RAMEND(15..8)};
      \node[below=of initsp1, block] (initsp2) {SP(15..8) \la RAMEND(15..8)};
      \node[subprogram,below=of initsp2]  (call init_ports) {INIT\_PORTS};
      \node[block,below=of call init_ports] (ldop) {operation \la PINB};

      \draw[line] (main) -- (initsp1);
      \draw[line] (initsp1) -- (initsp2);
      \draw[line] (initsp2) -- (call init_ports);
      \draw[line] (call init_ports) -- (ldop);

      \node[decision,below=of ldop] (op==0) {operation == 0?};
      \node[decision,below=of {op==0}] (op==1) {operation == 1?};
      \node[decision,below=of {op==1}] (op==2) {operation == 2?};


      \draw[line] (ldop) -- node[left] {No} (op==0);
      \draw[line] (op==0) -- node[left] {No} (op==1);
      \draw[line] (op==1) -- node[left] {No} (op==2);

      \node[on grid,cloud,right=of {op==0}] (sum) {Sum};
      \node[on grid,cloud,right=of {op==1}] (sub) {Subtract};
      \node[on grid,cloud,right=of {op==2}] (div) {Divide};
      \node[on grid,cloud,below=of div] (fact) {Factorial};

      \draw[line] (op==0) -- node[above] {Yes} (sum);
      \draw[line] (op==1) -- node[above] {Yes} (sub);
      \draw[line] (op==2) -- node[above] {Yes} (div);
      \draw[line] (op==2) |- node[above right] {No} (fact);

      \node[block,above right=10cm and 3cm of sum.center] (lds1) {summand1 \la PINA};
      \node[block,below=3ex of lds1] (lds2) {summand2 \la PINC};
      \node[block,below=3ex of lds2] (add) {summand1 \la summand1 + summand2};
      \node[block,below=3ex of add] (addout) {PORTD \la summand1};

      \draw[line] (sum.east) -- ++(1em,0) |- (lds1);
      \draw[line] (lds1) -- (lds2);
      \draw[line] (lds2) -- (add);
      \draw[line] (add) -- (addout);

      \node[block,above right=7cm and 3cm of sub.center] (ldsubtr) {subtrahend \la PINA};
      \node[block,below=3ex of ldsubtr] (ldminu) {minuend \la PINC};
      \node[block,below=3ex of ldminu] (subtract) {subtrahend \la subtrahend - minuend};
      \node[block,below=3ex of subtract] (subout) {PORTD \la subtrahend};

      \draw[line] (sub.east) -- ++(1em,0) |- (ldsubtr);
      \draw[line] (ldsubtr) -- (ldminu);
      \draw[line] (ldminu) -- (subtract);
      \draw[line] (subtract) -- (subout);

      \node[block,above right=4cm and 3cm of div.center] (ldnum) {numerator \la PINA};
      \node[block,below=3ex of ldnum] (lddenom) {denominator \la PINC};
      \node[subprogram,below=3ex of lddenom] (call divide) {DIVIDE};
      \node[block,below=3ex of call divide] (divout) {PORTD \la quotient};

      \draw[line] (div.east) -- ++(3em,0) |- (ldnum);
      \draw[line] (ldnum) -- (lddenom);
      \draw[line] (lddenom) -- (call divide);
      \draw[line] (call divide) -- (divout);

      \node[block,above right=1cm and 3cm of fact.center] (ldn) {n \la PINA};
      \node[subprogram,below=3ex of ldn] (call fact) {FACTORIAL};
      \node[block,below=3ex of call fact] (factout) {PORTD \la result};

      \draw[line] (fact.east) -- ++(1em,0) |- (ldn);
      \draw[line] (ldn) -- (call fact);
      \draw[line] (call fact) -- (factout);

      \node[cloud,right=of lddenom] (term) {TERM};

      \draw[line] (addout.east) -| (term);
      \draw[line] (subout.east) -| (term);
      \draw[line] (divout.east) -| (term);
      \draw[line] (factout.east) -| (term);

      \path[-latex']
      (term) edge[loop left,-latex'] ();
   \end{tikzpicture}
\end{center}

\subex{Assemblerprogramm}
\lstinputlisting[label={lst:calc}]{calculator.asm}

\end{document}
